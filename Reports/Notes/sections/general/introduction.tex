\section{Introduction}


Given observations of random variables $(X,Y)$, we suppose that there exist another set of random variables $Z$ that we do not observe,
yet that characterize $(X,Y)$ conditionally to $Z$.
$Z$ is then called a latent variable, or a hidden variable. \\

For instance, if we observe the weights of a given population through $X$, and that we aim at inferring their height $Y$, knowing the
sex of each individual through $Z$ could improve our predictions on $Y$.
Hence, assuming that there exist a latent variable to a given model adds structure to the model while improving the explainability,
as $Z$ characterizes the behavior of our dataset.
Typically, clustering methods like KMeans or Gaussian Mixtures Models (GMM) provide a discrete $Z$ given the observations, which are
interesting in the sense that they provide a categorical representation of our data. \\

However, finding such $Z$ given the observations is not straight forward, and not all dataset respond to a latent process, and may be not even a discrete one. \\

During this research internship, we aim at exploring latent models with discrete latent space in order to analyze the microbiota structure.
Our first focus will be on various methods to conceive latent models like Expectation-Maximization and variational models.


