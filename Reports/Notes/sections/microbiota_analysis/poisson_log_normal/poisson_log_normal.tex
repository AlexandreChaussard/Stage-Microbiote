\subsection{Poisson log-normal models}

\subsubsection{Introduction to PLN}

Abundance data are quite complex to model by nature, as they account both for the presence of a given entity in a
giving ecosystem and the given amount of the said entity in the studied ecosystem.
Extensive studies have been conducted on abundance data, among which we decide to focus on a state-of-art approach,
the Poisson Log-Normal (PLN) model introduced in \cite{PLN_chiquet}.

The PLN model belongs to the family of latent models, as it models the abundance of the entities using a Poisson distribution
that is parameterized by a latent gaussian.
Formally speaking, if $X \in \mathbb{R}^{n \times d}$ denotes an abundance matrix of $d$ entities among $n$ individuals, we get the following model:
\begin{itemize}
    \item $Z_i \sim \mathcal{N}(0, \Sigma)$ where $\Sigma \in \mathbb{R}^{d \times d}$ is to be learnt.
    \item $X_{ij} \sim \mathcal{P}(e^{Z_{ij}})$
    \item The $Z_i$ are independent, and $X_{ij}$ is independent from all other abundance conditionally to $Z_{ij}$.
\end{itemize}

This modelization turns out to have some great mathematical properties, among which one can highlight the faithful correlation, which means that
the correlation between $X_{ij}$ and $X_{ik}$ has the same sign as the correlation between $Z_{ij}$ and $Z_{ik}$.
Hence, if the correlation is $0$ in the latent space between two entities, we can deduce that the abundance of the two species are always independent.
This faithful correlation property then enables us to determine a latent interaction graph between entities, and therefore unveil hidden functional groups in our study.

\medskip

Furthermore, assuming we have $p$ covariates at our disposal in addition to the abundance for each individual, denoting such covariates by $D \in \mathbb{R}^{n \times p}$,
we can account for fixed effects such as clinical variables, or environmental variables.
Implementing it in the model is then done through the mean of the latent layer, assuming that the mean for the sample $i$ is then given by $\mu_i = D_i^\top \theta$, where $\theta$ is to be learnt.

Finally, one can add an offset to the model to model sampling efforts between sites of interest, or individuals (omitted in our specific case by assumption).

\medskip

While the modelization appears interpretable and straight-forward, the training of PLN models requires variational inference due to the latent layer in the problem.
Indeed, partially observed models generally make a great use of the E-M algorithm to estimate the parameters of the model, yet the E-step requires to compute $p(Z|X)$
which is not explicit in the general PLN framework.

\medskip

Instead, the authors of \cite{PLN_chiquet} suggest a variational approximation of the E-step.
Basically, we introduce a gaussian parameterized by $\Phi$, $q_{\Phi}(Z|X) \sim \prod_{k=1}^d \mathcal{N}(m_k, s_k)$, where the $m_k$ and $s_k$ are to be learnt (they compose $\Phi$).
Before performing the estimation step $h+1$, we then find the optimal variational approximation of the posterior given by
$$\Phi^{h+1} = arg\max_{\Phi} ELBO(\theta^h, \Phi)$$

As the ELBO turns out to be explicit in the independent gaussian variational approximation (see appendix of \cite{PLN_chiquet}).
Then, we perform the E-step by replacing the true posterior $p(Z|X)$ by $q_{\Phi^{h+1}}(Z|X)$, then we perform the M-step to compute $\theta^{h+1}$, and so on up until convergence.

\medskip

This model has proven to be very efficient to model independent samples, in very different context like PCA, Mixture Models, Network inference, for which the respective ELBO varies slightly.
The application of interest for our problem is the Network inference, which would enable us to unveil functional interactions between the bacteria of the gut microbiome.
The Network inference problem also comes with a sparsity-informed architecture suggested by the authors, by introducing a LASSO penalty on the covariance matrix $\Sigma$ of the latent layer.
Each connected component then forms a functional group in our interpretation. 

\subsubsection{Layer independent PLN}

Recall our specific microbiota framework, we introduce a taxonomy $\mathcal{T}$, with $N$ nodes, $L$ layers, and $K_{\ell}$ entities at layer $\ell$.
Let $(X_i)_{1 \leq i \leq n}$ the taxa-abundance samples of $n$ individuals, supported on the taxonomy $\mathcal{T}$.
We denote by $x_k^{(\ell)} \in \mathbb{N}$ the abundance of the $k$-th entity in a taxa-abundance sample at layer $\ell$.

\medskip

The layer independent PLN model would then consist of building one PLN model for each layer of the taxonomy.
Note beforehand that this model can create non-physical samples, as it is not informed by the previous layer to perform a normalization.
The main interest of this model is then to obtain a first functional graph of the entities within the taxonomy without developing external tools, but only using the PLN package.
Next sections will focus on new models derivated from PLN.

\medskip

For each layer $\ell$, introducing a latent variable $Z^{(\ell)}$, the formal model is such that $(Z^{(\ell)}, X^{(\ell)}) \sim PLN(0, \Sigma^{(\ell)})$  where $\Sigma^{(\ell)} \in \mathbb{R}^{K_{\ell} \times K_{\ell}}$.
We then have to learn $L$ covariance matrices, which can be seen as solving $L$ PLN problems independently due to the layer independence between the $(Z^{(\ell)}, X^{(\ell)})$.

\subsubsection{Multi-layer covariance PLN}

Assume the same framework as previously.
This time we introduce dependency between the layers by making some assumptions on the joint distribution of the $Z^{(\ell)}$, so that:
\begin{itemize}
    \item $Z = (Z^{(1)}, \dots, Z^{(L)}) \sim \mathcal{N}(0, \Sigma)$, where $\Sigma \in \mathbb{R}^{(K_1 + \dots + K_{L})^2}$
    \item $Z^{(\ell)} \sim \mathcal{N}(0, \Sigma^{(\ell)})$ where $\Sigma^{(\ell)} \in \mathbb{R}^{K_{\ell}^2}$
    \item $x_k^{(\ell)} \sim \mathcal{P}(e^{Z_k^{(\ell)}})$
\end{itemize}

This model differs from the following by the join distribution assumption structure that replaces the independence per layer.
The consequence is that we are now able to model functional groups between all entities in the taxonomy, leading to inter layer functional group rather than just intra layer.

\medskip

Using a variational approximation, this model turns out to be tractable (see proposition \ref{proposition:VEM_PLN_multilayer_objective}).