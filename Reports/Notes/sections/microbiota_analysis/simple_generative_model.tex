\subsection{Simple generative model: no latent variable}

\subsubsection{Context and objective}

In first approximation, we would like to define a generative model that does not exploit any latent structure.
Such model, parameterized by $\theta$, aims at finding an optimal distribution in the sense of the maximum of likelihood,
within a family of distributions yet to be defined.
The maximum likelihood objective is given below as:
$$
\theta^* = arg\max_{\theta} p_{\theta}(X, T)
$$
One can rewrite the joint distribution as follow:
$$
\begin{align}
    p_{\theta}(X, T) &= \prod_{i=1}^n p_{\theta}(X_i, T_i) \\
                    &= \prod_{i=1}^n p_{\theta}(T_i) p_{\theta}(X_i | T_i)
\end{align}
$$

As a result, to compute this objective, we need to define a prior $p_{\theta}(T_i)$ that generates trees,
and a posterior distribution $p_{\theta}(X_i | T_i)$ that generates abundance data from a sampled tree.

\subsubsection{Design of the prior}

We aim at defining a parameterized distribution $p_{\theta}(T)$ from which one can sample trees.
We would also like this distribution to model the trees of the microbiota dataset, meaning that
the generated trees should look like the ones from the dataset as well, and respect the phylogenetic constraints. \\

Consequently, we introduce this first simple generative process to characterize $p_{\theta}(T)$:
\begin{algorithm}[H]
    \caption{Independent Branches Tree (IBT) sampling}
    \begin{algorithmic}
        \STATE Define the tree $T$
        \STATE Sample $K \sim \mathcal{B}(D, \pi)$ the number of leaf entities in the tree
        \For{$i \in \{1, \dots, K\}$}
            \STATE \quad Define the branch $b_i$
            \STATE \quad Set the root of the tree to the lowest precision level entity as $u_0^{(i)}$
            \STATE \quad Define $j = 0$
            \While{$u_{j}^{(i)}$ is not the empty node index}
                \STATE \quad \quad Append the node $u_{j}^{(i)}$ to the branch $b_i$
                \STATE \quad \quad Gather the possible children of the node $u_{j}^{(i)}$ as $V_{j}^{(i)}$
                \STATE \quad \quad Append the empty node index in $V_{j}^{(i)}$
                \STATE \quad \quad Sample $u_{j+1}^{(i)}$ following $g_{\pi_{u_{j}^{(i)}}}$ a discrete distribution over $V_{j}^{(i)}$ with weights $\pi_{u_{j}^{(i)}}$
                \STATE \quad \quad $j \leftarrow j + 1$
            \EndWhile
            \STATE \quad Add the branch $b_i$ to the tree $T$.
        \EndFor
        \RETURN{$T$}
    \end{algorithmic}
    \label{alg:algorithm}
\end{algorithm}

The IBT generates trees with independent branches, so that each node in a branch is dependent from the previous ones,
while each branch starts from the same root and are i.i.d.
Hence, such structure is one the most simple one could think of when generating trees. \\
Another step would be to add dependency between branches, for instance looking at each generation level what other entities
have been selected in all the branches at the same level before.
Though that would require a supplementary modelization step that we leave on the side for now. \\

Following the IBT modelization, we can compute the prior as:
$$
\begin{align}
    p_{\theta}(T) &= p_{\theta}(b_1, \dots, b_{|T|}) \\
                &= \prod_{i=1}^{|T|} p_{\theta}(b_i) \\
                &= \prod_{i=1}^{|T|} p_{\theta}(u_1^{(i)}, \dots, u_{|b_i|}^{(i)}) \\
                &= \prod_{i=1}^{|T|} \underbrace{p(u_1^{(i)})}_{1} p_{\pi_{u_1^{(i)}}}(u_2^{(i)} | u_1^{(i)}) \dots p_{\pi_{u_{|b_i|-1}^{(i)}}}(u_{|b_i|}^{(i)} | u_{|b_i|-1}^{(i)}) \\
                &= \prod_{i=1}^{|T|} \prod_{j=1}^{|b_i|-1} p_{\pi_{u_j^{(i)}}}(u_{j+1}^{(i)} | u_{j}^{(i)}) \\
                &= \prod_{i=1}^{|T|} \prod_{j=1}^{|b_i|-1} \pi_{u_j^{(i)} \rightarrow u_{j+1}^{(i)}} \right
\end{align}
$$

\subsubsection{Design of the posterior}

Now that we have a way to generate trees, we need to define an explicit stochastic relationship between the abundance
data and the structure of the tree.
Such inevitably exists, as when observing $T$, the abundance of an entity that isn't present in $T$ is necessarily 0.
Similarly, it is highly likely that when observing the presence of certain entities in $T$ that induces a high abundance of
another neighbour entity (interaction between bacteria). \\

For context, we recall that $X$ is a matrix of shape $(D, U)$.
We denote by $X^{(l)}$ the $l$-th line of the abundance matrix.

Hence, since we would like an explicit model here, we will design a posterior $p_{\theta}(X|T)$ which is markovian relatively to the layers of the tree:
$$
p_{\theta}(X^{(l)} | X^{(1:l-1)}, T) = p_{\theta}(X^{(l)} | X^{(l-1)}, T)
$$
We assume the following framework:
\begin{itemize}
    \item We denote $X^{(l)} = (x_l^{(1)}, \dots, x_{l}^{(U_l)})$ the abundance vector at layer $l$.
    \item $X^{(1)} = [1, 0, \dots, 0]$, since it's the root of the tree, only one entity gets the whole weight.
    \item We assume that for all $l \geq 2, X^{(l)} | T \sim \mathcal{D}(\alpha_l)$.
    \item Each value in $X^{(l)}$ is restricted by the following set of constraints:
        \begin{itemize}
              \item If node $k$ at layer $l-1$ has no child, then we create a ghost child for which the abundance value is the same as the parent.
                    This node should be removed at the end of the procedure.
                    This is an artificial way of keeping the Dirichlet modelisation, so that al rows sum to 1 while generating X.
              \item If node $k$ at layer $l-1$ has one child, then its abundance value is the same for the child node.
              \item If node $k$ at layer $l-1$ has at least two children, the children abundance sums to the parent's abundance value.
        \end{itemize}
        We denote by $\mathcal{C}_T$ the set of vectors that verify the previous constraints for the tree $T$.
        Hence, we obtain the following conditional distribution:
        $$
        \begin{align}
            p_{\gamma_l}(X^{(l+1)} | X^{(l)}, T) &= \frac{p_{\gamma_l}(X^{(l+1)}, X^{(l)}, T)}{p_{\alpha_l}(X^{(l)}, T)} \\
                                            &= \frac{p_{\alpha_{l+1}}(X^{(l+1)} | T)}{p_{\alpha_{l}}(X^{(l)} | T)} p(X^{(l)} | X^{(l+1)}, T) \\
                                            &= \frac{p_{\alpha_{l+1}}(X^{(l+1)} | T)}{p_{\alpha_{l}}(X^{(l)} | T)} \mathds{1}_{\mathcal{C}_T}\left(X^{(l)}, X^{(l+1)}\right)
        \end{align}
        $$
\end{itemize}

The whole posterior is then given by:
$$
\begin{align}
    p_{\theta}(X | T) &= \prod_{i=1}^n p_{\theta}(X_i | T_i) \\
                    &= \prod_{i=1}^n p_{\theta}(X_i^{(1)}, \dots, X_i^{(D)} | T_i) \\
                    &= \prod_{i=1}^n p(X_i^{(1)} | T_i) \prod_{l=1}^{D-1} p_{\gamma_j}(X_i^{(l+1)} | X_i^{(l)}, T_i) \\
                    &= \prod_{i=1}^n \mathds{1}_{X_i^{(1)} = e_1} \prod_{l=1}^{D-1} \frac{p_{\alpha_{l+1}}(X_i^{(l+1)} | T_i)}{p_{\alpha_{l}}(X_i^{(l)} | T_i)} \mathds{1}_{\mathcal{C}_{T_i}}\left(X_i^{(l)}, X_i^{(l+1)}\right) \\
                    &= \prod_{i=1}^n p_{\alpha_{D}}(X_i^{(D)} | T_i) \mathds{1}_{X_i^{(1)} = e_1} \prod_{l=1}^{D-1} \mathds{1}_{C_{T_i}}(X_i^{(l)} X_i^{(l+1)})
\end{align}
$$

Hence, to generate an abundance matrix $X$ out of a tree $T$, we just need to sample the highest precision layer of the tree
and roll it up to the top thanks to the tree structure that we observe.
As a result, this posterior is only characterized by one parameter $\alpha_D$ that is the dirichlet parameter of the ultimate layer of the tree.

\subsubsection{Optimization of the objective}

Recall the optimization objective, written under the maximum of the log-likelihood:
$$
\begin{align}
    \theta^* &= arg \max_{\theta} \log p_{\theta}(X,T) \\
            &= arg \max_{\theta} \sum_{i=1}^n \log p_{\theta}(X_i, T_i) \\
            &= arg \max_{\theta} \sum_{i=1}^n \log p_{\theta}(T_i) + \log p_{\theta}(X_i | T_i)
\end{align}
$$

Notice that the first term of the loss is made only of the prior, while the second one is made only of the posterior.
Since these two distributions do not share common parameters in $\theta$, we can proceed to optimize the parameters of each distribution
without taking care of the other term. \\

\newcommand{\transitionproba}{\pi_{a \rightarrow b}}
\newcommand{\transitionprobasum}{\pi_{u_j^{(k)} \rightarrow u_{j+1}^{(k)}}}
\newcommand{\transitionbranch}{b_{a \rightarrow b}^{(i)}}
\newcommand{\children}{\mathcal{C}}
\newcommand{\lagrangian}{\mathcal{L}}

Starting with the prior term, it only depends on the transition probabilities denoted as $(\transitionproba)_{(a,b) \in T}$
where $(a,b) \in T$ denotes a parent-to-child path.
To compute $\transitionproba^*$, we introduce the set of branches that contain the path $a \rightarrow b$ for given tree $T_i$:
$$
\transitionbranch = \{b \in T_i | (a,b) \in b\}
$$
For a given node $a$ we also denote by $\children(a)$ the set of possible children from $a$.
Note that $\children(a)$ is never empty, as every node can lead to stopping the branch, which means it contains the ghost empty node.

The optimization objective for $\transitionproba$ is given by:
$$
\begin{equation}
    \begin{aligned}
        \transitionproba^* = arg \max_{\transitionproba} \quad & \sum_{i=1}^n \sum_{k=1}^{|T_i|} \sum_{j=1}^{|b_i| - 1} \log \transitionprobasum \\
        \textrm{s.t.} \quad & \sum_{l \in \children(a)} \pi_{a \rightarrow l} = 1\\
        & \transitionproba \geq 0    \\
    \end{aligned}
    \label{eq:prior_objective}
\end{equation}
$$

Such optimization objective is convex, with convex constraints that are solvable through Lagrange duality.
Computing the lagrangian we obtain:
$$
\lagrangian(\transitionproba, \lambda) = \sum_{i=1}^n \sum_{k=1}^{|T_i|} \sum_{j=1}^{|b_i| - 1} \log \transitionprobasum - \lambda_1 \transitionproba + \lambda_2 \left(\sum_{l \in \children(a)} \pi_{a \rightarrow l} - 1\right) + \iota_{\mathbb{R}^{+}}(\lambda_1, \lambda_2)
$$

The KKT conditions are then written as:
$$
\begin{align}
    0 &\in \partial_{\transitionproba} \lagrangian(\transitionproba, \lambda) \\
    \lambda_2 \left(\sum_{l \in \children(a)} \pi_{a \rightarrow l} - 1\right) &= 0 \\
    \lambda_1 &\geq 0 \\
    -\lambda_1 \transitionproba &\leq 0 \\
    -\lambda_1 \transitionproba &= 0 \\
\end{align}
$$

Using the first equality, we obtain:
$$
\transitionproba = \frac{1}{\lambda_1 - \lambda_2} \sum_{i=1}^{n} |\transitionbranch|
$$

Using the sum over all $l \in \children(a)$ on the previous result, assuming $\lambda_2 = 0$, the second KKT identity gives us:
$$
\sum_{l \in \children(a)} \pi_{a \rightarrow l} = \sum_{l \in \children(a)} \frac{\sum_{i=1}^n |\transitionbranch|}{\lambda_1} = 1
$$

Hence, we obtain that $\lambda_1 = \sum_{l \in \children(a)} \sum_{i=1}^n |\transitionbranch| = n_a$, the number of branches that goes through $a$.
Finally, we obtain the optimal transition probability:
$$
\fbox{
    \displaystyle\transitionproba^* = \frac{1}{n_a} \sum_{i=1}^n |\transitionbranch|
}
$$
One can verify that the KKT conditions are then satisfied as well. \\

Now, we compute the optimal parameter for the posterior that is given by $\alpha_D^* = (\alpha_D^{(1)}, \dots, \alpha_D^{(U)})$.
The optimization objective for each component is the following:
$$
(\alpha_D^{(j)})^* = arg \max_{\alpha_D^{(k)}} \sum_{i=1}^n \log p_{\alpha_D}(X_i^{(D)}|T_i) + \log \mathds{1}_{X_i^{(1)} = e_1} + \sum_{k=1}^{D-1} \log \mathds{1}_{C_{T_i}}(X_i^{(l)}, X_i^{(l+1)})
$$
Since the indicator functions do not depend on the parameter, we end up on a known problem of maximum of likelihood for Dirichlet distribution \cite{dirichlet_digamma_trick}.
As we have done the maths in another context for the dirichlet mixture, we get a similar result that ends up on the following fixed point algorithm:
$$
\fbox{
    \displaystyle\left(\alpha_D^{(j)}\right)^{(m+1)} = \psi^{-1}\left(\frac{1}{n} \sum_{i=1}^n \log \left[X_i^{(D)}\right]_j + \psi \left(\sum_{r=1}^U \left(\alpha_{D}^{(r)}\right)^{(m)}\right) \right)
}
$$

We now have a way to compute $\theta^*$ as a MLE of the parameter of the joint distribution.