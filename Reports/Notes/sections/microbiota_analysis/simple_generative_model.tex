\subsection{Simple generative model: no latent variable}

\subsubsection{Context and objective}

In first approximation, we would like to define a generative model that does not exploit any latent structure.
Such model, parameterized by $\theta$, aims at finding an optimal distribution in the sense of the maximum of likelihood,
within a family of distributions yet to be defined.
The maximum likelihood objective is given below as:
$$
\theta^* = arg\max_{\theta} p_{\theta}(X, T)
$$
One can rewrite the joint distribution as follow:
$$
\begin{align}
    p_{\theta}(X, T) &= \sum_{i=1}^n p_{\theta}(X_i, T_i) \\
                    &= \sum_{i=1}^n p_{\theta}(T_i) p_{\theta}(X_i | T_i)
\end{align}
$$

As a result, to compute this objective, we need to define a prior $p_{\theta}(T_i)$ that generates trees,
and a posterior distribution $p_{\theta}(X_i | T_i)$ that generates abundance data from a sampled tree.

\subsubsection{Design of the prior}

We aim at defining a parameterized distribution $p_{\theta}(T)$ from which one can sample trees.
We would also like this distribution to model the trees of the microbiota dataset, meaning that
the generated trees should look like the ones from the dataset as well, and respect the phylogenetic constraints. \\

Consequently, we introduce this first simple generative process to characterize $p_{\theta}(T)$:
\begin{algorithm}[H]
    \caption{Independent Branches Tree (IBT) sampling}
    \begin{algorithmic}
        \STATE Define the tree $T$
        \STATE Sample $K \sim \mathcal{B}(D, \pi)$ the number of leaf entities in the tree
        \For{$i \in \{1, \dots, K\}$}
            \STATE \quad Define the branch $b_i$
            \STATE \quad Set the root of the tree to the lowest precision level entity as $u_0^{(i)}$
            \STATE \quad Define $j = 0$
            \While{$u_{j}^{(i)}$ is not the empty node index}
                \STATE \quad \quad Append the node $u_{j}^{(i)}$ to the branch $b_i$
                \STATE \quad \quad Gather the possible children of the node $u_{j}^{(i)}$ as $V_{j}^{(i)}$
                \STATE \quad \quad Append the empty node index in $V_{j}^{(i)}$
                \STATE \quad \quad Sample $u_{j+1}^{(i)}$ following $g_{\pi_{u_{j}^{(i)}}}$ a discrete distribution over $V_{j}^{(i)}$ with weights $\pi_{u_{j}^{(i)}}$
                \STATE \quad \quad $j \leftarrow j + 1$
            \EndWhile
            \STATE \quad Add the branch $b_i$ to the tree $T$.
        \EndFor
        \RETURN{$T$}
    \end{algorithmic}
    \label{alg:algorithm}
\end{algorithm}

The IBT generates trees with independent branches, so that each node in a branch is dependent from the previous ones,
while each branch starts from the same root and are i.i.d.
Hence, such structure is one the most simple one could think of when generating trees. \\
Another step would be to add dependency between branches, for instance looking at each generation level what other entities
have been selected in all the branches at the same level before.
Though that would require a supplementary modelization step that we leave on the side for now. \\

Following the IBT modelization, we can compute the prior as:
$$
\begin{align}
    p_{\theta}(T) &= p_{\theta}(b_1, \dots, b_{|T|}) \\
                &= \prod_{i=1}^{|T|} p_{\theta}(b_i) \\
                &= \prod_{i=1}^{|T|} p_{\theta}(u_1^{(i)}, \dots, u_{|b_i|}^{(i)}) \\
                &= \prod_{i=1}^{|T|} \underbrace{p(u_1^{(i)})}_{1} p_{\pi_{u_1^{(i)}}}(u_2^{(i)} | u_1^{(i)}) \dots p_{\pi_{u_{|b_i|-1}^{(i)}}}(u_{|b_i|}^{(i)} | u_{|b_i|-1}^{(i)}) \\
                &= \prod_{i=1}^{|T|} \prod_{j=1}^{|b_i|-1} p_{\pi_{u_j^{(i)}}}(u_{j+1}^{(i)} | u_{j}^{(i)}) \\
                &= \prod_{i=1}^{|T|} \prod_{j=1}^{|b_i|-1} \pi_{u_j^{(i)} \rightarrow u_{j+1}^{(i)}} \right
\end{align}
$$

\subsubsection{Design of the posterior}

Now that we have a way to generate trees, we need to define an explicit stochastic relationship between the abundance
data and the structure of the tree.
Such inevitably exists, as when observing $T$, the abundance of an entity that isn't present in $T$ is necessarily 0.
Similarly, it is highly likely that when observing the presence of certain entities in $T$ that induces a high abundance of
another neighbour entity (interaction between bacteria). \\

Hence, since we would like an explicit model here, we will design a posterior $p_{\theta}(X|T)$ at each precision level
using a parametric distributions like the Dirichlet distribution. We assume the following framework:
\begin{itemize}
    \item We denote $X^{(l)} = (x_l^{(1)}, \dots, x_{l}^{(U_l)})$ the abundance vector at layer $l$.
    \item $X^{(1)} = (1)$ since it's the root of the tree.
    \item $X^{(2)} \sim \mathcal{D}(\alpha_2)$
    \item $\forall l \geq 3, X^{(l)} \sim \mathcal{D}(\alpha_l)$ if there's more than one node in the layer.
        Each value in $X^{(l)}$ is constrained by the following set of constraint $\mathcal{C}$:
        \begin{itemize}
              \item If node $k$ at layer $l-1$ has one child, then its abundance value is the same for the child node.
              \item If node $k$ at layer $l-1$ has at least two children, the children abundance sums to the parent's abundance value.
        \end{itemize}
        Hence, we obtain the following conditional distribution:
        $$
        \begin{align}
            p_{\gamma}(X^{(l+1)} | X^{(l)}) &= \frac{p_{\gamma}(X^{(l+1)}, X^{(l)})}{p_{\alpha_l}(X^{(l)})} \\
                                            &= \frac{p_{\alpha_{l+1}}(X^{(l+1)})}{p_{\alpha_{l}}(X^{(l)})} p(X^{(l)} | X^{(l+1)}) \\
                                            &= \frac{p_{\alpha_{l+1}}(X^{(l+1)})}{p_{\alpha_{l}}(X^{(l)})} \mathbbm{1}_{\mathcal{C}}
        \end{align}
        $$
\end{itemize}