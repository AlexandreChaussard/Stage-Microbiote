\subsubsection{Design of the prior}

% Design

We aim at defining a parameterized distribution $p_{\theta}(T)$ from which one can sample trees.
We would also like this distribution to model the trees of the microbiota dataset, meaning that
the generated trees should look like the ones from the dataset as well, and respect the phylogenetic constraints. \\

\newcommand{\nodeindexsetgeneric}{\mathcal{A}_T^{(L)}}
\newcommand{\treeconstrainedindicator}{\mathds{1}_T(T^{(l)} \leftarrow T^{(l+1)})}

Consequently, we introduce a first simple generative process to characterize $p_{\theta}(T)$ that we would call the independent nodes tree generator.
Before describing the generation method, let us introduce the framework of it:
\begin{itemize}
    \item Describe $T$ as a succession of $L$ layers: $T = (T^{(1)}, \dots, T^{(L)})$.
          We assume that we have no missing data, so to say, all the leaves the tree are reaching the precision layer $L$.
    \item Describe a given layer $l$ as a discrete vector in $\{0, 1\}^{U}$.
          To each possible node at layer $l$ we can associate an index $k$ so that we denote the nodes by $u_k^{(l)}$.
          A node $u_k^{(l)}$ is activated if it is valued as $1$ in $T^{(l)}$, otherwise it is not.
    \item Denote by $\mathcal{A}_T^{(l)}$ the set of indexes $k \in \{1, \dots, U\}$ so that
          the node $u_k^{(l)}$ is activated in the tree $T$ at layer $l$, i.e., $u_k^{(l)} = 1$.
    \item Denote by $\treeconstrainedindicator$ the constrained indicator function of the tree $T$ between layers $l$ and $l+1$.
          This constrained indicator function ensures each node in $T^{(l+1)}$ has an activated parent in $T^{(l)}$, so the graph of the tree $T$ is respected.
\end{itemize}

Now that the framework is clear and defined, we describe the Independent Nodes Tree generative process:
\begin{itemize}
    \item The root node of the tree is deterministic, since all trees begin to the same root ancestor.
          Hence, we have:
          $$
            p(T^{(1)}) = \mathds{1}_{T^{(1)} = e_1}
          $$
    \item For all $l \geq 2$, $k \in \{1, \dots, U\}$, $u_k^{(l)} \sim \mathcal{B}(\pi_{k}^{(l)})$ a Bernoulli parameterized by $\pi_{k}^{(l)} \in [0,1]$.
          For now, we make the major assumption that all nodes are independent within a given layer.
    \item The dependency between the layers is markovian:
         $$
            p_{\theta}(T^{(l+1)} | T^{(1:l)}) = p_{\theta}(T^{(l+1)} | T^{(l)})
         $$
\end{itemize}

Noting these framework properties, we can describe the distribution of such prior model on the trees:
$$
\begin{align}
    p_{\theta}(T) &= p_{\theta}\left(T^{(1)}, \dots, T^{(L)}\right) \\
                    &= p\left(T^{(1)}\right) \prod_{l=1}^{L-1} p_{\theta}\left(T^{(l+1)}|T^{(l)}\right) \\
                    &= \mathds{1}_{T^{(1)} = e_1} \prod_{l=1}^{L-1} \frac{p_{\theta}\left(T^{(l+1)}\right) p_{\theta}\left(T^{(l)}|T^{(l+1)}\right)}{p_{\theta}\left(T^{(l)}\right)} \\
                    &= \mathds{1}_{T^{(1)} = e_1} p_{\theta}\left(T^{(L)}\right) \prod_{l=1}^{L-1} \treeconstrainedindicator \\
                    &= \mathds{1}_{T^{(1)} = e_1} p_{\theta}\left(\bigcap_{k \in \nodeindexsetgeneric} \{u_k^{(L)} = 1\}, \bigcap_{k' \notin \nodeindexsetgeneric} \{u_{k'}^{(L)} = 0\}\right) \prod_{l=1}^{L-1} \treeconstrainedindicator \\
                    &= \mathds{1}_{T^{(1)} = e_1} \prod_{k \in \nodeindexsetgeneric} \pi_k^{(L)} \prod_{k' \notin \nodeindexsetgeneric}(1 - \pi_{k'}^{(L)}) \prod_{l=1}^{L-1} \treeconstrainedindicator
\end{align}
$$

Looking at the formula we obtain, we notice that generating a tree can be done by sampling nodes at the highest precision layer $L$
using independent Bernoulli distribution for each node.
Then, using the trees constraints that we know, we can roll up the tree and activate the parent nodes of the generated children, and so on until we reach the root.
Hence, the only parameters of the prior are the Bernoulli parameters of each node, denoted $\pi_k^{(L)}$ for the node indexed by $k$. \\

% Optimization
Now that the prior of the trees is well defined, we would like to compute an optimal value of
$\pi^{(L)} = \left(\pi_1^{(L)}, \dots, \pi_{U}^{(L)}\right)$ in the sense of the maximum of likelihood. \\
The maximum log-likelihood objective for $\pi_k^{(L)}$ over our microbiota dataset can then be written as:
$$
\begin{equation}
    \begin{aligned}
        \left(\pi_j^{(L)}\right)^* = arg \max_{\pi_j^{(L)}} \quad & \sum_{i=1}^n \sum_{k \in \mathcal{A}^{(L)}_{T_i}} \log \pi_k^{(L)} + \sum_{k' \notin \mathcal{A}^{L}_{T_i}}\log (1 - \pi_{k'}^{(L)}) \\
        \textrm{s.t.} \quad & \forall k, \pi_{k}^{(L)} \in [0, 1] \\
    \end{aligned}
    \label{eq:prior_transition_objective}
\end{equation}
$$

This objective corresponds to the maximum log-likelihood estimator of a Bernoulli distribution,
which can be solved to obtain the following estimator:
$$
\fbox{
    \displaystyle
    \left(\pi_j^{(L)}\right)^* = \frac{1}{n} \sum_{i=1}^n \mathds{1}_{j \in \mathcal{A}^{(L)}_{T_i}}
}
$$