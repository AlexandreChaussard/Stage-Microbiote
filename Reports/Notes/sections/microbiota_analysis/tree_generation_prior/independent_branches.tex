\subsubsection{Design of the prior and maximum likelihood estimator}

% Design

We aim at defining a parameterized distribution $p_{\theta}(T)$ from which one can sample trees.
We would also like this distribution to model the trees of the microbiota dataset, meaning that
the generated trees should look like the ones from the dataset as well, and respect the phylogenetic constraints. \\

Consequently, we introduce this first simple generative process to characterize $p_{\theta}(T)$:
\begin{algorithm}[H]
    \caption{Independent Branches Tree (IBT) sampling}
    \begin{algorithmic}
        \STATE Define the tree $T$
        \STATE Sample $K \sim \beta$ the number of leaf entities in the tree
        \For{$i \in \{1, \dots, K\}$}
            \STATE \quad Define the branch $b_i$
            \STATE \quad Set the root of the tree to the lowest precision level entity as $u_0^{(i)}$
            \STATE \quad Define $j = 0$
            \While{$u_{j}^{(i)}$ is not the empty node index}
                \STATE \quad \quad Append the node $u_{j}^{(i)}$ to the branch $b_i$
                \STATE \quad \quad Gather the possible children of the node $u_{j}^{(i)}$ as $V_{j}^{(i)}$
                \STATE \quad \quad Append the empty node index in $V_{j}^{(i)}$
                \STATE \quad \quad Sample $u_{j+1}^{(i)}$ following $g_{\pi_{u_{j}^{(i)}}}$ a discrete distribution over $V_{j}^{(i)}$ with weights $\pi_{u_{j}^{(i)}}$
                \STATE \quad \quad $j \leftarrow j + 1$
            \EndWhile
            \STATE \quad Add the branch $b_i$ to the tree $T$.
        \EndFor
        \RETURN{$T$}
    \end{algorithmic}
    \label{alg:algorithm}
\end{algorithm}

The IBT generates trees with independent branches, so that each node in a branch is dependent from the previous ones,
while each branch starts from the same root and are i.i.d.
Hence, such structure is one the most simple one could think of when generating trees. \\
Another step would be to add dependency between branches, for instance looking at each generation level what other entities
have been selected in all the branches at the same level before.
Though that would require a supplementary modelization step that we leave on the side for now. \\

Following the IBT modelization, we can compute the prior as:
$$
\begin{align}
    p_{\theta}(T) &= p_{\theta}(K = |T|, b_1, \dots, b_{K}) \\
    &= p_{\beta}(K=|T|) \prod_{i=1}^{|T|} p_{\theta}(b_i) \\
    &= \beta_{|T|} \prod_{i=1}^{|T|} p_{\theta}(u_1^{(i)}, \dots, u_{|b_i|}^{(i)}) \\
    &= \beta_{|T|} \prod_{i=1}^{|T|} \underbrace{p(u_1^{(i)})}_{1} p_{\pi_{u_1^{(i)}}}(u_2^{(i)} | u_1^{(i)}) \dots p_{\pi_{u_{|b_i|-1}^{(i)}}}(u_{|b_i|}^{(i)} | u_{|b_i|-1}^{(i)}) \\
    &= \beta_{|T|} \prod_{i=1}^{|T|} \prod_{j=1}^{|b_i|-1} p_{\pi_{u_j^{(i)}}}(u_{j+1}^{(i)} | u_{j}^{(i)}) \\
    &= \beta_{|T|} \prod_{i=1}^{|T|} \prod_{j=1}^{|b_i|-1} \pi_{u_j^{(i)} \rightarrow u_{j+1}^{(i)}} \right
\end{align}
$$

% Optimization

\newcommand{\transitionproba}{\pi_{a \rightarrow b}}
\newcommand{\transitionprobasum}{\pi_{u_j^{(k)} \rightarrow u_{j+1}^{(k)}}}
\newcommand{\transitionbranch}{b_{a \rightarrow b}^{(i)}}
\newcommand{\children}{\mathcal{C}}
\newcommand{\lagrangian}{\mathcal{L}}

Now that the likelihood of the prior is well defined, let's try to compute a maximum likelihood estimator of such distribution.
Recall that the prior is defined by the discrete distribution $\beta$ and the transition probabilities denoted as $(\transitionproba)_{(a,b) \in T}$
where $(a,b) \in T$ denotes a parent-to-child path. \\

The maximum log-likelihood objective for $\beta^*$ is given by:
$$
\begin{equation}
    \begin{aligned}
        \beta^*_k = arg \max_{\beta_k} \quad & \sum_{i=1}^n \log \beta_{|T_i|} \\
        \textrm{s.t.} \quad & \forall i, \beta \geq 0 \\
        \quad & \sum_{i=1}^D \beta_i = 1
    \end{aligned}
    \label{eq:prior_beta_objective}
\end{equation}
$$

The previous optimization problem is rapidly solved through Lagrange duality, so that we obtain a straight-forward estimator:
$$
\fbox{
    \displaystyle \beta_k^* = \frac{1}{n} \sum_{i=1}^n \mathds{1}_{|T_i| = k}
}
$$

To compute $\transitionproba^*$, we introduce the set of branches that contain the path $a \rightarrow b$ for a given tree $T_i$:
$$
\transitionbranch = \{b \in T_i | (a,b) \in b\}
$$
For a given node $a$ we also denote by $\children(a)$ the set of possible children from $a$.
Note that $\children(a)$ is never empty, as every node can lead to stopping the branch, which means it contains the ghost empty node.

The optimization objective for $\transitionproba$ is given by:
$$
\begin{equation}
    \begin{aligned}
        \transitionproba^* = arg \max_{\transitionproba} \quad & \sum_{i=1}^n \sum_{k=1}^{|T_i|} \sum_{j=1}^{|b_i| - 1} \log \transitionprobasum \\
        \textrm{s.t.} \quad & \sum_{l \in \children(a)} \pi_{a \rightarrow l} = 1\\
        & \transitionproba \geq 0    \\
    \end{aligned}
    \label{eq:prior_transition_objective}
\end{equation}
$$

Such optimization objective is convex, with convex constraints that are solvable through Lagrange duality.
Computing the lagrangian we obtain:
$$
\lagrangian(\transitionproba, \lambda) = \sum_{i=1}^n \sum_{k=1}^{|T_i|} \sum_{j=1}^{|b_i| - 1} \log \transitionprobasum - \lambda_1 \transitionproba + \lambda_2 \left(\sum_{l \in \children(a)} \pi_{a \rightarrow l} - 1\right) + \iota_{\mathbb{R}^{+}}(\lambda_1, \lambda_2)
$$

The KKT conditions are then written as:
$$
\begin{align}
    0 &\in \partial_{\transitionproba} \lagrangian(\transitionproba, \lambda) \\
    \lambda_2 \left(\sum_{l \in \children(a)} \pi_{a \rightarrow l} - 1\right) &= 0 \\
    \lambda_1 &\geq 0 \\
    -\lambda_1 \transitionproba &\leq 0 \\
    -\lambda_1 \transitionproba &= 0 \\
\end{align}
$$

Using the first equality, we obtain:
$$
\transitionproba = \frac{1}{\lambda_1 - \lambda_2} \sum_{i=1}^{n} |\transitionbranch|
$$

Using the sum over all $l \in \children(a)$ on the previous result, assuming $\lambda_2 = 0$, the second KKT identity gives us:
$$
\sum_{l \in \children(a)} \pi_{a \rightarrow l} = \sum_{l \in \children(a)} \frac{\sum_{i=1}^n |\transitionbranch|}{\lambda_1} = 1
$$

Hence, we obtain that $\lambda_1 = \sum_{l \in \children(a)} \sum_{i=1}^n |\transitionbranch| = n_a$, the number of branches that goes through $a$.
Finally, we obtain the optimal transition probability:
$$
\fbox{
    \displaystyle\transitionproba^* = \frac{1}{n_a} \sum_{i=1}^n |\transitionbranch|
}
$$
One can verify that the KKT conditions are then satisfied as well.