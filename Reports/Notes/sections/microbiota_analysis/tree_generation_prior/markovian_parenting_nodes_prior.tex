\subsubsection{Design of the prior}

% Design

We aim at defining a parameterized distribution $p_{\theta}(T)$ from which one can sample trees.
We would also like this distribution to model the trees of the microbiota dataset, meaning that
the generated trees should look like the ones from the dataset as well, and respect the phylogenetic constraints. \\

\newcommand{\nodeparent}{\mathcal{P}}

Consequently, we introduce a first simple generative process to characterize $p_{\theta}(T)$ that we would call the markovian parenting tree generator.
Before describing the generation method, let us introduce the framework of it:
\begin{itemize}
    \item Describe $T$ as a succession of $L$ layers: $T = (T^{(1)}, \dots, T^{(L)})$.
    We assume that we have no missing data, so to say, all the leaves the tree are reaching the precision layer $L$.
    \item Describe a given layer $\ell$ as a discrete vector in $\{0, 1\}^{U}$.
    To each possible node at layer $\ell$ we can associate an index $k$ so that we denote the nodes by $u_k^{(\ell)} \in \{0, 1\}$.
    $$T^{(\ell)} = (u_1^{(\ell)}, \dots, u_{K_{\ell}}^{(\ell)})$$
    A node $u_k^{(\ell)}$ is activated if it is valued as $1$ in $T^{(\ell)}$, otherwise it is not.
    \item We introduce the function $\nodeparent$ that takes a node $u_k^{(\ell)}$ as input, and output the parent of the node if it is well defined in $T$,
          otherwise it outputs $0$ as if the parent was not activated:
            $$
            \nodeparent(u_k^{(\ell)}) = \left\{
            \begin{array}{ll}
                1 & \mbox{if } \text{parent of $u_k^{(\ell)}$ exists and is activated}\\
                0 & \mbox{otherwise}
            \end{array}
            \right.
            $$
\end{itemize}

Now that the framework is clear and defined, we describe the generative process:
\begin{itemize}
    \item The root node of the tree is deterministic, since all trees begin to the same root ancestor.
    Hence, we have:
    $$
    p(T^{(1)}) = \delta_{e_1}(T^{(1)})
    $$

    \item For all $l \geq 2, k \in \{1, \dots, U\}$, we assume that the activation of the node $u_k^{(\ell)}$
        is distributed as a Bernoulli conditionally to its parent activation, parameterized by $\pi_k^{(\ell)}$:
        $$
        u_k^{(\ell)} | \left \{\nodeparent(u_k^{(\ell)}) = 1 \right\} \sim \mathcal{B}(\pi_{k}^{(\ell)})
        $$
        To respect the tree architecture, we assume that if $\nodeparent(u_k^{(\ell)}) = 0$ then the probability for the children
        to be activated is deterministic and set to $0$, as a child can not exist without his parent.
    \item For now, we make the major assumption that all nodes are independent within a given layer conditionally to their parents, and they only depend on their respective parent, so that:
    $$
    \begin{align}
        p\left(u_1^{(\ell)}, \dots, u_{K_{\ell}}^{(\ell)} | \nodeparent(u_1^{(\ell)}), \dots, \nodeparent(u_{K_{\ell}}^{(\ell)}) \right) &= \prod_{k=1}^{K_{\ell}} p\left(u_k^{(\ell)} | \nodeparent(u_k^{(\ell)})\right)
    \end{align}
    $$

    \item The dependency between the layers of the tree is markovian:
    $$
    p_{\theta}(T^{(\ell+1)} | T^{(1:\ell)}) = p_{\theta}(T^{(\ell+1)} | T^{(\ell)})
    $$
\end{itemize}

Noting these framework properties, we can describe the log-likehood of such prior model on the trees using proposition \ref{proposition:bernoulli_tree_prior}:
$$
\begin{align}
    p_{\theta}(T_1, \dots, T_n) = \sum_{i=1}^n \sum_{i=1}^{L-1} \sum_{k=1}^{K_{\ell}} \nodeparent(u_k^{(\ell+1)}) \left[u_k^{(\ell+1)} \log \pi_k^{(\ell+1)} + (1 - u_k^{(\ell+1)}) \log (1 - \pi_k^{(\ell+1)})  \right]
\end{align}
$$

Looking at the previous formula, we obtain that such prior is parameterized by the activation probabilities $\pi_k^{(\ell)}$ of each node $u_k^{(\ell)}$.
Furthermore, due to the indicator function expressed through $\nodeparent(u_k^{(\ell+1)})$, the update of a given activation probability
will only be impacted by the trees which have the node $u_k^{(\ell)}$ in any branch. \\

% Optimization
Now that the prior of the trees is well defined, we would like to compute an optimal value of
$\pi^{(\ell)} = \left(\pi_1^{(\ell)}, \dots, \pi_{U}^{(\ell)}\right)$ in the sense of the maximum of likelihood.
Using proposition \ref{proposition:mle_bernoulli_tree_prior}, we obtain
$$
\fbox{
    \displaystyle
    \left(\pi_k^{(\ell)}\right)^* &= \frac{\sum_{i=1}^n \nodeparent(u_{k,i}^{(\ell)}) u_{k,i}^{(\ell)}}{\sum_{i=1}^n \nodeparent(u_{k,i}^{(\ell)})}
}
$$

This estimator is actually the common MLE for a Bernoulli parameter estimation, except that it limits
the computation of the estimation to all trees that respect the root constraint, and that could possess the node $u_k^{(\ell)}$
since they must have the parent node $\nodeparent(u_k^{(\ell)})$.