\subsubsection{Design of the posterior and maximum likelihood estimator}

\newcommand{\childrennode}{\mathcal{C}}

% Design of the posterior

Now that we have a way to generate trees, we need to define an explicit stochastic relationship between the abundance
data and the structure of the tree.
Such inevitably exists, as when observing $T$, the abundance of an entity that isn't present in $T$ is necessarily 0.
Similarly, it is highly likely that when observing the presence of certain entities in $T$ that induces a high abundance of
another neighbour entity (interaction between bacteria). \\

For context, we recall that $X$ is a matrix of shape $(L, U)$,
where $L$ is the maximum precision level of the trees (assumed to be the same for all trees for now) and $U$ the maximum number of entities at each depth of the trees.
We denote by $X^{(l)}$ the $l$-th line of the abundance matrix, for which up to $U_l$ elements should be non-zero. \\

We assume the following framework:
\begin{itemize}
    \item Let $X = (X^{1}, \dots, X^{(L)})$ a given abundance matrix associated to a tree $T$.
    \item We denote $X^{(l)} = (x_1^{(l)}, \dots, x_{K_l}^{(l)})$ the abundance vector at layer $l$.
    \item For an abundance node $x_k^{(l)}$, we denote by $\childrennode(x_k^{(l)})$ the set of abundance children associated to that node.
    \item $X^{(1)} = [1, 0, \dots, 0]$, since it's the root of the tree, only one entity gets the whole weight.
    \item Since we would like a simple explicit model at first, we design a posterior $p_{\theta}(X|T)$ which is markovian relatively to the layers of the tree,
            so that the abundance at the next layer are only impacted by the previous layers abundance for now:
            $$
            p_{\theta}(X^{(l)} | X^{(1:l-1)}, T) = p_{\theta}(X^{(l)} | X^{(l-1)}, T^{(l)})
            $$
    \item In addition to the markovianity, we assume that at a given layer, the abundance only depends of their respective parent and their siblings:
    $$p(X^{(l+1)} | X^{(l)}, T^{(l+1)}) = \prod_{k=1}^{K_l} p(\childrennode(x_k^{(l)}) | x_k^{(l)}, T^{(l+1)})$$
    \item Each value in $X^{(l)}$ is restricted by the following set of constraints due to the nature of the abundance in a tree structure:
            \begin{itemize}
                \item If node $k$ at layer $l-1$ has one child, then its abundance value is the same for the child node.
                \item If node $k$ at layer $l-1$ has at least two children, the children abundance sums to the parent's abundance value.
            \end{itemize}
    \item Since we deal with proportions in abundance vectors, it seems natural to use the Dirichlet distribution in first assumption.
          Hence, we assume that for all $l \geq 2$, if $|\childrennode(x_k^{(l)})| > 1$,
            $$\childrennode(x_k^{(l+1)}) | x_k^{(l)}, T \sim x_k^{(l)} \mathcal{D}(\alpha_k^{(l)} \odot T^{(l+1)})$$
          We denote by $f_{\alpha_k^{(l)} \odot T^{(l+1)}}$ the density of this distribution, parameterized by $\alpha_k^{(l)} \odot T^{(l+1)}$ which denotes
          a masked version of $\alpha_k^{(l)}$ relatively to the activated nodes of the tree at $T^{(l)}$.
          Notice that setting this distribution framework enables us to verify the constraints given above by renormalizing the layer with the obtained weights.
          Naturally, if $|\childrennode(x_k^{(l)})| = 1$, we have $\childrennode(x_k^{(l)}) \sim \delta_{x_k^{(l)}}\left(\childrennode(x_k^{(l)})\right)$ to respect the constraints.
\end{itemize}

Noting the previous framework, the whole abundance distribution conditionally to a tree $T$ can be written as (see lemma \ref{lemma:abundance_posterior_bernoulli_tree}):
$$
p(X|T) = \delta_{e_1}(X^{(1)}) \prod_{l=1}^{L-1} \prod_{k=1}^{K_l} \left(
        \delta_{x_k^{(l)}}\left(\childrennode(x_k^{(l)})\right)^{\mathds{1}_{|\childrennode(x_k^{(l)})| = 1}}
        \frac{1}{x_k^{(l)}} f_{\alpha_k^{(l)} \odot T^{(l+1)}} \left(\frac{\childrennode(x_k^{(l))}}{x_k^{(l)}}\right)^{\mathds{1}_{|\childrennode(x_k^{(l)})| > 1}}
        \right)
$$

% Optimization
One can then compute a maximum likelihood estimator of each $\alpha_k^{(l)}$ using a fixed point iterative algorithm that is described
in the appendix as lemma \ref{MLE_abundance_bernoulli_tree}:
$$
\fbox{
    \displaystyle
    \alpha_{k,v}^{(l)} \leftarrow \psi^{-1} \left(\frac{\sum_{i=1}^n \mathds{1}_{|\childrennode(x_{k,i}^{(l)})| > 1} \mathds{1}_{v \in \mathcal{V}(\alpha_k^{(l)}, T_i)} \left[ \psi\left(\sum_{u \in \mathcal{V}(\alpha_k^{(l)}, T_i)} \alpha_{k,u}^{(l)}\right) + \log \frac{\left[\childrennode(x_{k,i}^{(l)})\right]_v}{x_{k,i}^{(l)}} \right]}
    {\sum_{i=1}^n \mathds{1}_{|\childrennode(x_{k,i}^{(l)})| > 1} \mathds{1}_{v \in \mathcal{V}(\alpha_k^{(l)}, T_i)}} \right) \right)
}
$$