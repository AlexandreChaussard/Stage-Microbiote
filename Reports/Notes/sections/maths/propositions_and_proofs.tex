\section{Appendix}

\subsection{Microbiota analysis}

\begin{lemma}(Bernoulli tree prior)
    \label{lemma:bernoulli_tree_prior}
    \\
    Let $T = (T^{(1)}, \dots, T^{(L)})$ a random tree of depth $L$. \\
    For all $l \in \{1, \dots, L\}$, denote $T^{(l)} = (u_1^{(l)}, \dots, u_{K_l}^{(l)})$ the nodes at layer $l$,
    such that $u_k^{(l)} \in \{0, 1\}$ denotes whether or not the node is activated (1 if activated). \\
    Denote $\mathcal{P}(u_k^{(l)})$ the parent of the node $u_k^{(l)}$ that outputs $1$ if the parent exists and is activated, $0$ otherwise. \\
    Assume that:
    \begin{itemize}
        \item $p(T^{(1)}) = \delta_{e_1}(T^{(1)})$
        \item $\forall l \geq 2, p(T^{(l+1)} | T^{(1:l)}) = p(T^{(l+1)} | T^{(l)})$
        \item $\forall l \geq 2, k \in \{1, \dots, K_l\}, u_k^{(l)} | \left\{\mathcal{P}(u_k^{(l)}) = 1 \right\} \sim \mathcal{B}(\pi_k^{(l)}) \right$
        \item $\displaystyle p\left(u_1^{(l)}, \dots, u_{K_l}^{(l)} | \mathcal{P}(u_1^{(l)}), \mathcal{P}(u_{K_l}^{(l)})\right) = \prod_{k=1}^{K_l} p\left(u_k^{(l)} | \mathcal{P}(u_k^{(l)})\right)$
    \end{itemize}

    Then, the distribution of $T$ can be written as:
    $$
    p(T) = \delta_{e_1}(T^{(1)}) \prod_{l=1}^{L-1} \prod_{k=1}^{K_l} \left(\left(\pi_k^{(l+1)}\right)^{u_k^{(l+1)}} \left(1-\pi_k^{(l+1)}\right)^{1 - u_k^{(l+1)}} \right)^{\mathcal{P}(u_k^{l+1})}
    $$
\end{lemma}

\begin{proof}
    $$
    \begin{align}
        p(T) &= p\left(T^{(1)}, \dots, T^{(L)}\right) \\
        &= p\left(T^{(1)}\right) \prod_{l=1}^{L-1} p\left(T^{(l+1)}|T^{(l)}\right) \\
        &= \delta_{e_1}(T^{(1)}) \prod_{l=1}^{L-1} p\left(u_1^{(l+1)}, \dots, u_{K_l}^{(l+1)} | \nodeparent(u_1^{(l+1)}), \dots, \nodeparent(u_{K_l}^{(l+1)})\right) \\
        &= \delta_{e_1}(T^{(1)}) \prod_{l=1}^{L-1} \prod_{k=1}^{K_l} p\left(u_k^{(l+1)} | \nodeparent(u_k^{(l+1)}) \right) \\
        &= \delta_{e_1}(T^{(1)}) \prod_{l=1}^{L-1} \prod_{k=1}^{K_l} \left( \pi_k^{(l+1)} u_k^{(l+1)} + \left(1 - \pi_k^{(l+1)} \right)(1 - u_k^{(l+1)}) \right)^{\nodeparent(u_k^{(l+1)})} \\
        &= \delta_{e_1}(T^{(1)}) \prod_{l=1}^{L-1} \prod_{k=1}^{K_l} \left( \left(\pi_k^{(l+1)}\right)^{u_k^{(l+1)}} \left(1 - \pi_k^{(l+1)} \right)^{1 - u_k^{(l+1)}} \right)^{\nodeparent(u_k^{(l+1)})}
    \end{align}
    $$
\end{proof}

\begin{lemma}[MLE of the Bernoulli tree prior]
    \label{lemma:mle_bernoulli_tree_prior}
    \\
    Recall the context of \ref{lemma:bernoulli_tree_prior}. \\
    Let $(T_1, \dots, T_n)$ $n$ trees i.i.d following the bernoulli tree prior.
    Denote the maximum likelihood objective by
    $$
    \begin{equation}
        \begin{align}
            arg \max_{\pi_j^{(m)}} \quad & \sum_{i=1}^n \sum_{l=1}^{L-1} \sum_{k=1}^{K_l} \nodeparent(u_{k,i}^{(l+1)}) \left[u_{k,i}^{(l+1)} \log \pi_{k}^{(l+1)} + (1 - u_{k,i}^{(l+1)}) \log (1 - \pi_k^{(l+1)}) \right] \\
            \textrm{s.t.} \quad & \forall k,l, \pi_{k}^{(l)} \in [0, 1] \\
        \end{align}
        \label{eq:prior_transition_objective}
    \end{equation}
    $$

    Then the maximum likelihood estimator is given by
    $$
    \left(\pi_k^{(l)}\right)^* = \frac{\sum_{i=1}^n \nodeparent(u_{k,i}^{(l)}) u_{k,i}^{(l)}}{\sum_{i=1}^n \nodeparent(u_{k,i}^{(l)})}
    $$
\end{lemma}

\begin{proof}
    Simply by deriving the objective we obtain:
    $$
    \begin{align}
        \partial_{\pi_j^{(m)}} \log p(T_1, \dots, T_n) &= \sum_{i=1}^n \nodeparent(u_{j,i}^{(m)}) \left[ u_{j,i}^{(m)} \frac{1}{\pi_j^{(m)}} + (1 - u_{j,i}^{(m)}) \frac{-1}{1-\pi_j^{(m)}}\right] \\
        &= \frac{1}{\pi_j^{(m)}} \sum_{i=1}^n \nodeparent(u_{j,i}^{(m)}) u_{j,i}^{(m)} - \frac{1}{1 - \pi_j^{(m)}} \sum_{i=1}^n \nodeparent(u_{j,i}^{(m)}) (1 - u_{j,i}^{(m)})
    \end{align}
    $$

    Looking for $0$ valued gradient, we end up with:
    $$
    \begin{align}
    (1 - \pi_j^{(m)}) \sum_{i=1}^n \nodeparent(u_{j,i}^{(m)}) u_{j,i}^{(m)} - \pi_j^{(m)} \sum_{i=1}^n \nodeparent(u_{j,i}^{(m)}) (1 - u_{j,i}^{(m)}) &= 0 \\
    \pi_j^{(m)} \sum_{i=1}^n \nodeparent(u_{j,i}^{(m)}) &= \sum_{i=1}^n \nodeparent(u_{j,i}^{(m)}) u_{j,i}^{(m)} \\
    \pi_j^{(m)} &= \frac{\sum_{i=1}^n \nodeparent(u_{j,i}^{(m)}) u_{j,i}^{(m)}}{\sum_{i=1}^n \nodeparent(u_{j,i}^{(m)})}
    \end{align}
    $$

    Since the obtained value respects the constraint, that concludes the proof.
\end{proof}

\begin{lemma}[Law of $aX$]
    \label{lemma:law_of_aX}
    \\
    Let $X \sim f(x)$ a random variable with density $f$. \\
    Let $a \in (0, 1)$. \\
    The distribution of $aX$ is given by:
    $$p_{aX}(u) = \frac{1}{a} f\left(\frac{u}{a}\right)$$
\end{lemma}

\begin{proof}
    We use the transfer theorem.
    Let $g$ a continuous and measurable function.
    $$
    \begin{align}
        \mathbb{E}[g(aX)] &= \int g(ax) f(x) dx \\
                          &= \int g(u) f\left(\frac{u}{a}\right) \frac{1}{a} du
    \end{align}
    $$
    Hence, one can identify the distribution of $aX$ as $p_{aX}(u) = \frac{1}{a} f\left(\frac{u}{a}\right)$
\end{proof}

\begin{lemma}[Abundance distribution conditionally to the trees]
    \label{lemma:abundance_posterior_bernoulli_tree}
    \\
    Let $X = (X^{(1)}, \dots, X^{(L)})$ the abundance matrix of a tree $T$ with depth $L$. \\
    For all $l \in \{1, \dots, L\}$, denote $X^{(l)} = (x_1^{(l)}, \dots, x_{K_l}^{(l)})$ the abundance value of each node of the tree.
    Denote $\childrennode(x_k^{(l)})$ the vector of children abundances related to $x_k^{(l)}$, which is empty if it has no children.
    Assume that:
    \begin{itemize}
        \item $p(X^{(1)}) = \delta_{e_1}(X^{(1)})$
        \item $p(X^{(l+1)} | X^{(1:l)}, T) = p(X^{(l+1)} | X^{(l)}, T^{(l+1)})$
        \item $p(X^{(l+1)} | X^{(l)}, T^{(l+1)}) = \displaystyle\prod_{k=1}^{K_l} p(\childrennode(x_k^{(l)}) | x_k^{(l)}, T^{(l+1)})$
        \item For all $l \in \{2, \dots, L\}, k \in \{1, \dots, K_l\}$,
              \begin{itemize}
                  \item If $|\childrennode(x_k^{(l)})| > 1$: $\childrennode(x_k^{(l)}) | x_k^{(l)}, T^{(l+1)} \sim x_k^{(l)} D(\alpha_k^{(l)} \odot T^{(l+1)})$
                  \item If $|\childrennode(x_k^{(l)})| = 1$, $\childrennode(x_k^{(l)}) |  x_k^{(l)}, T^{(l+1)} \sim \delta_{x_k^{(l)}}\left(\childrennode(x_k^{(l)})\right)$
              \end{itemize}
    \end{itemize}

    Then, the distribution of $X$ conditionally to $T$ is given by:
    $$
        p(X|T) = \delta_{e_1}(X^{(1)}) \prod_{l=1}^{L-1} \prod_{k=1}^{K_l} \left(
    \delta_{x_k^{(l)}}\left(\childrennode(x_k^{(l)})\right)^{\mathds{1}_{|\childrennode(x_k^{(l)})| = 1}}
    \frac{1}{x_k^{(l)}} f_{\alpha_k^{(l)} \odot T^{(l+1)}} \left(\frac{\childrennode(x_k^{(l)})}{x_k^{(l)}}\right)^{\mathds{1}_{|\childrennode(x_k^{(l)})| > 1}}
                \right)
    $$
    
\end{lemma}

\begin{proof}
    $$
    \begin{align}
        p(X | T) &= p(X^{(1)}, \dots, X^{(L)} | T) \\
                &= p(X^{(1)} | T^{(1)}) \prod_{l=1}^{L-1} p(X^{(l+1)} | X^{(l)}, T^{(l+1)}) \\
                &= \delta_{e_1}(X^{(1)}) \prod_{l=1}^{L-1} \prod_{k=1}^{K_l} p(\childrennode(x_k^{(l)}) | x_k^{(l)}, T^{(l+1)}) \\
                &= \delta_{e_1}(X^{(1)}) \prod_{l=1}^{L-1} \prod_{k=1}^{K_l} \left(
                                                                                \delta_{x_k^{(l)}}\left(\childrennode(x_k^{(l)})\right)^{\mathds{1}_{|\childrennode(x_k^{(l)})| = 1}}
                                                                                \left[\underbrace{\frac{1}{x_k^{(l)}} f_{\alpha_k^{(l)} \odot T^{(l+1)}} \left(\frac{\childrennode(x_k^{(l)})}{x_k^{(l)}}\right)}_{\text{lemma \ref{lemma:law_of_aX}}}\right]^{\mathds{1}_{|\childrennode(x_k^{(l)})| > 1}}
                                                                            \right)
    \end{align}
    $$
\end{proof}

\begin{lemma}[MLE of the abundance distribution a posteriori]
    \label{MLE_abundance_bernoulli_tree}
    \\
    Consider the context of lemma \ref{lemma:abundance_posterior_bernoulli_tree}. \\
    We consider a data set $(X_i, T_i)_{1 \geq i \geq n}$ of abundance matrix and trees. \\
    Denote by $\mathcal{V}(\alpha_k^{(l)}, T) = \left\{v \in \mathbb{N} | \alpha_{k,v}^{(l)} \in \alpha_k^{(l)} \odot T^{(l+1)}\right\}$
    the set of indexes so that the coordinate of $\alpha_k^{(l)}$ is not masked by $T^{(l+1)}$. \\
    The maximum likelihood estimator of the distribution characterising the abundance conditionally to the tree is then
    given by the following fixed point algorithm: \\

    $
    \forall l \in \{1, \dots, L\}, k \in \{1, \dots, K_l\}, v \in \{1, \dots, |\alpha_k^{(l)}|\}
    $,

    $$
    \alpha_{k,v}^{(l)} \leftarrow \psi^{-1} \left(\frac{\sum_{i=1}^n \mathds{1}_{|\childrennode(x_{k,i}^{(l)})| > 1} \mathds{1}_{v \in \mathcal{V}(\alpha_k^{(l)}, T_i)} \left[ \psi\left(\sum_{u \in \mathcal{V}(\alpha_k^{(l)}, T_i)} \alpha_{k,u}^{(l)}\right) + \log \frac{\left[\childrennode(x_{k,i}^{(l)})\right]_v}{x_{k,i}^{(l)}} \right]}
    {\sum_{i=1}^n \mathds{1}_{|\childrennode(x_{k,i}^{(l)})| > 1} \mathds{1}_{v \in \mathcal{V}(\alpha_k^{(l)}, T_i)}} \right) \right)
    $$
\end{lemma}

\begin{proof}
    The maximum likelihood objective can be written as:
    $$
    \begin{align*}
        arg \max_{\alpha_{j,v}^{(m)}} \quad & \sum_{i=1}^n \sum_{l=1}^{L-1} \sum_{k=1}^{K_l} \mathds{1}_{|\childrennode(x_{k,i}^{(l)})| > 1} \log f_{\alpha_k^{(l)} \odot T_i^{(l+1)}} \left(\frac{C(x_{k,i}^{(l)})}{x_{k,i}^{(l)}} \right) \\
    \end{align*}
    $$
    Using the Dirichlet distribution expression, we can write the following:
    \small
    $$
    \begin{align}
        \log f_{\alpha_k^{(l)} \odot T_i^{(l+1)}}\left(\frac{C(x_{k,i}^{(l)})}{x_{k,i}^{(l)}} \right) = \log \Gamma \left(\sum_{v \in \mathcal{V}(\alpha_k^{(l)}, T_i)} \alpha_{k,v}^{(l)}\right)
        - \sum_{v \in \mathcal{V}(\alpha_k^{(l)}, T_i)} \Gamma \left( \alpha_{k,v}^{(l)} \right)
        + \sum_{v \in \mathcal{V}(\alpha_k^{(l)}, T_i)} \left(\alpha_{k,v}^{(l)} - 1\right) \log \frac{\left[\childrennode(x_{k,i}^{(l)})\right]_v}{x_{k,i}^{(l)}}
    \end{align}
    $$
    \normalsize

    Writing $\psi$ the digamma function, the derivative of the objective relatively to a fixed $\alpha_{k,v}^{(l)}$ is given by:
    $$
    \partial_{\alpha_{k,v}^{(l)}} \log p(X|T) = \sum_{i=1}^n \mathds{1}_{|\childrennode(x_{k,i}^{(l)})| > 1} \mathds{1}_{v \in \mathcal{V}(\alpha_k^{(l)}, T_i)} \left[\psi\left(\sum_{u \in \mathcal{V}(\alpha_k^{(l)}, T_i)} \alpha_{k,u}^{(l)} \right) - \psi\left(\alpha_{k,v}^{(l)} \right) + \log \frac{\left[\childrennode(x_{k,i}^{(l)})\right]_v}{x_{k,i}^{(l)}}\right]
    $$

    Looking for $0$ valued gradient, we obtain the following equation:
    $$
    \psi\left(\alpha_{k,v}^{(l)}\right) = \frac{\sum_{i=1}^n \mathds{1}_{|\childrennode(x_{k,i}^{(l)})| > 1} \mathds{1}_{v \in \mathcal{V}(\alpha_k^{(l)}, T_i)} \left[ \psi\left(\sum_{u \in \mathcal{V}(\alpha_k^{(l)}, T_i)} \alpha_{k,u}^{(l)}\right) + \log \frac{\left[\childrennode(x_{k,i}^{(l)})\right]_v}{x_{k,i}^{(l)}} \right]}
                                                {\sum_{i=1}^n \mathds{1}_{|\childrennode(x_{k,i}^{(l)})| > 1} \mathds{1}_{v \in \mathcal{V}(\alpha_k^{(l)}, T_i)}} \right)
    $$

    Using the reference trick from \cite{dirichlet_digamma_trick}, one can compute an iterative fixed point algorithm to solve the previous equation,
    leading to:
    $$
    \alpha_{k,v}^{(l)} \leftarrow \psi^{-1} \left(\frac{\sum_{i=1}^n \mathds{1}_{|\childrennode(x_{k,i}^{(l)})| > 1} \mathds{1}_{v \in \mathcal{V}(\alpha_k^{(l)}, T_i)} \left[ \psi\left(\sum_{u \in \mathcal{V}(\alpha_k^{(l)}, T_i)} \alpha_{k,u}^{(l)}\right) + \log \frac{\left[\childrennode(x_{k,i}^{(l)})\right]_v}{x_{k,i}^{(l)}} \right]}
    {\sum_{i=1}^n \mathds{1}_{|\childrennode(x_{k,i}^{(l)})| > 1} \mathds{1}_{v \in \mathcal{V}(\alpha_k^{(l)}, T_i)}} \right) \right)
    $$
\end{proof}