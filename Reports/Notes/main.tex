\documentclass[12pt]{article}
\usepackage[utf8]{inputenc}
\usepackage[english]{babel}
\usepackage[T1]{fontenc}
\usepackage{amsmath}
\usepackage{amsfonts}
\usepackage{dsfont}
\usepackage{setspace}
\usepackage[dvipsnames]{xcolor}
\usepackage{graphicx}
\usepackage{mathtools}
\usepackage{float}
\usepackage{numprint}
\usepackage{url}
\usepackage{dirtree}
\usepackage{algorithmic}
\usepackage{algorithm}
\usepackage{amsthm}
\newtheorem{theorem}{Theorem}[section]
\newtheorem{corollary}{Corollary}[theorem]
\newtheorem{proposition}[theorem]{Proposition}
\newtheorem{lemma}[theorem]{Lemma}
\usepackage[hidelinks]{hyperref}
\hypersetup{
    linktoc=all,     %set to all if you want both sections and subsections linked
}

\usepackage[Glenn]{fncychap}
\usepackage{enumitem}
\usepackage{multicol}
\pagestyle{plain}
\usepackage{subfig}
\usepackage{siunitx}

% You can set the margine of the text by setting the top, bottom, left and right properties. Also, if using more than one columns, you can change the columnsep property to set the seperation space between two columns.
\usepackage[a4paper,top=30mm,bottom=30mm,columnsep=6mm,left=20mm,right=20mm]{geometry}

% uncommnet and change the following two commands to set seperation line width between coulmns.

% \setlength{\columnsep}{1cm}
% \setlength{\columnseprule}{0.6pt}

%########################## Paper Meta data#########################
% You can change the variables in the following section to set them globally thrghout the whole paper.

% Title of the paper
\newcommand{\papertitle}{Internship notes}
\newcommand{\shorttitle}{}

% Full name of the author(s)
\newcommand{\authorname}{Alexandre CHAUSSARD}

% Module code
\newcommand{\acmodulecode}{01/05/2023}

% Module name
\newcommand{\acmodulename}{}

% Name of the college
\newcommand{\accollegename}{Laboratoire de Probabilité, Statistiques et Modélisation}

% Date
\newcommand{\wrdate}{May 2023}
%########################## Paper Meta data#########################

\usepackage{fancyhdr}

\usepackage{biblatex}
\usepackage{algpseudocode}
\usepackage{textcomp}
\bibliographystyle{ieeetr}
\addbibresource{library.bib}

\pagestyle{fancy}
\fancyhead{}
\fancyhead[L]{\textsc{\shorttitle}}
\fancyfoot{}
\fancyfoot[R]{\thepage}
\fancyfoot[L]{\acmodulecode\  \textsc{\acmodulename}}
% \fancyfoot[C]{\thepage}
\renewcommand{\headrulewidth}{0.6pt}
\renewcommand{\footrulewidth}{0.6pt}


\title{\papertitle}
\author{\authorname}


\begin{document}
    
% general section    
\newcommand{\HRule}{\rule{\linewidth}{0.5mm}} 		
\begin{titlepage}
    \begin{center}
        %\vspace*{1cm}

        \center 
        \begin{figure}[H]
            \centering
            \includegraphics[scale=.4]{images/logo_lpsm}
            \label{fig:logo_lab}
        \end{figure}
        
        % Document info
        \textsc{Laboratoire de Probabilité, Statistiques et Modélisation}\\[0.5cm]
        \textsc{\large May 2023}\\[2cm] 
        
        % Course Code
        \HRule \\[1cm]
        { \huge \bfseries Internship report notes}\\[1cm]			
        \HRule \\[.5cm]
        \large

        \vspace*{1cm}
        
        \emph{Master 2 - Data Science} \\[1cm]

        Alexandre CHAUSSARD  \\ [0.5cm]
    \end{center}
\end{titlepage}

\newpage
\section{Results to check and/or review}

List of results that would require a \color{red}\textbf{review}\color{black}:
\begin{itemize}
    \item Proposition \ref{proposition:MLE_abundance_bernoulli_tree}: check the validity of the fixed point algorithm part using the provided reference.
\end{itemize}

\newpage
\tableofcontents
\newpage

\section{Introduction}


Given observations of random variables $(X,Y)$, we suppose that there exist another set of random variables $Z$ that we do not observe,
yet that characterize $(X,Y)$ conditionally to $Z$.
$Z$ is then called a latent variable, or a hidden variable. \\

For instance, if we observe the weights of a given population through $X$, and that we aim at inferring their height $Y$, knowing the
sex of each individual through $Z$ could improve our predictions on $Y$.
Hence, assuming that there exist a latent variable to a given model adds structure to the model while improving the explainability,
as $Z$ characterizes the behavior of our dataset.
Typically, clustering methods like KMeans or Gaussian Mixtures Models (GMM) provide a discrete $Z$ given the observations, which are
interesting in the sense that they provide a categorical representation of our data. \\

However, finding such $Z$ given the observations is not straight forward, and not all dataset respond to a latent process, and may be not even a discrete one. \\

During this research internship, we aim at exploring latent models with discrete latent space in order to analyze the microbiota structure.
Our first focus will be on various methods to conceive latent models like Expectation-Maximization and variational models.



\newpage

% EM
\section{Expectation-Maximization}

\subsection{Overview}

The Expectation-Maximization (EM) algorithm, first introduced in \cite{expectation_maximization_source}, is vast class of latent models.
It is based on the following smart decomposition of the data:
$$
\log p_{\theta}(X) = \underbrace{\mathbb{E}_{p_{\widehat{\theta}}(Z|X)}[\log p_{\theta}(X,Z)|X]}_{Q(\widehat{\theta}, \theta)} - \mathbb{E}_{p_{\widehat{\theta}}(Z|X)}[\log p_{\theta}(Z|X)|X]
$$

The idea behind this decomposition is that $\log p_{\theta}(X)$ is generally not tractable since it's an integral, while the complete likelihood $p_{\theta}(X,Z)$ is generally manageable.
Note that since $\log p_{\theta}(X)$ can not be computed, $\log p_{\theta}(Z|X)$ can't either by extension.
Hence, we introduce $\log p_{\widehat{\theta}}(Z|X)$ where $\widehat{\theta}$ is the maximum likelihood estimator of $\theta$:
$$
\widehat{\theta} = arg\max_{\theta} \log p_{\theta}(X)
$$

The main trick of the EM algorithm relies in the idea that $Q(\widehat{\theta}, \theta)$ is sufficient to compute a maximum likelihood estimator of $\theta$.
Indeed, consider the following algorithm:
\begin{algorithm}[H]
    \caption{Expectation-Maximization}
    \begin{algorithmic}
        \REQUIRE $\widehat{\theta}$
        \STATE Repeat until convergence
            \STATE \quad \textbf{Expectation}: compute $p_{\widehat{\theta}}(Z|X)$ to compute $Q(\widehat{\theta}, .)$
            \STATE \quad \textbf{Maximization}: $\widehat{\theta} = arg\max_{\theta} Q(\widehat{\theta}, \theta)$
    \end{algorithmic}

    \textbf{Output:} $\widehat{\theta}$

    \label{alg:em_algo}
\end{algorithm}

If we denote by $\widehat{\theta}^h$ the iterates of this algorithm, one can show using Jensen's inequality that:
$$
\log p_{\widehat{\theta}^{h+1}}(X) \geq \log p_{\widehat{\theta}^h}(X)
$$

As a result, the EM algorithm maximizes the likelihood, producing an estimator $\widehat{\theta}$ that is an MLE of $\theta$.
Note that we don't have a convergence certainty towards the best maximizer of the likelihood, only to a local maxima.
Hence, the EM algorithm is heavily sensitive to the initialization we pick. \\

All is required now is to choose a latent model so we can perform the EM algorithm, meaning that we have to define the distributions of the followings:
\begin{itemize}
    \item $Z \sim \mathcal{B}(K, \pi)$, conveniently set to a binomial of parameter $\pi$ so that it's discrete and simple to manage.
    \item $X|Z=k \sim p_{\gamma(k)}(X|Z=k)$, which is where we have the most choice to make.
\end{itemize}

In such models, $\theta = (\pi, \gamma(0), ..., \gamma(K))$.
In the next section, we will study a specific fork architecture of the gaussian mixture case.
\subsection{Gaussian Mixture Linear Classifier}

\subsubsection{Framework and computations}
Consider the case for which we observe $(X_i, Y_i)_{1 \leq i \leq n}$ i.i.d samples, where $X_i$ denotes a feature vector and $Y_i$ a label in a classification framework.
We aim at introducing a linear classifier that exploits a latent structure over $(X, Y)$, so that we have the following latent model:
\begin{itemize}
    \item $Z_i \sim \mathcal{B}(K, \pi)$, we note $\pi_k = \mathbb{P}(Z_i = k)$
    \item $X_i|Z_i=k \sim \mathcal{N}(\mu_k, \sigma_k I)$, we denote by $f_k(X_i)$ its density.
    \item $\mathbb{P}(Y_i = 1 | X_i, Z_i=k) = \sigma(W_{e,k}^T e_k + W_{x,k}^T X_i) = p_k(X_i)$, where $e_k$ denotes a vector from the canonical basis of $\mathbb{R}^K$.
\end{itemize}

As we want to perform the EM algorithm find an MLE of
$$
\theta = (\pi, \mu_1, ..., \mu_K, \sigma_1, ..., \sigma_K, W_{e,1}, ..., W_{e,K}, W_{x,1}, ..., W_{x,K})
$$
we start by computing the \textbf{expectation} step by assessing $p_{\widehat{\theta}}(Z_i = k|X_i,Y_i)$, using Bayes rules ($\hat{\pi}, \hat{p}, \hat{f}$ signify that we evaluate these quantities using the current estimate $\widehat{\theta}$):
$$
\begin{align}
    p_{\widehat{\theta}}(Z_i = k|X_i, Y_i) &= \frac{\hat{\pi}_k \hat{f}_k(X_i) \left(Y_i \hat{p}_k(X_i) + (1 - Y_i) (1 - \hat{p}_k(X_i)) \right)}{\sum_{j=1}^K \hat{\pi}_j \hat{f}_j(X_i) \left(Y_i \hat{p}_j(X_i) + (1 - Y_i) (1 - \hat{p}_j(X_i)) \right)}
    \\&= \tau_{ik}
\end{align}
$$

Now, we can safely evaluate $Q(\widehat{\theta}, \theta)$ for any $\theta$:
$$
\begin{align}
    Q(\widehat{\theta}, \theta) &= \mathbb{E}_{p_{\widehat{\theta}}(Z|X)}[\log p_{\theta}(X, Y, Z) | X] \\
    &= \sum_{i=1}^n \sum_{k=0}^K \log p_{\theta}(X_i, Y_i, Z_i = k) \tau_{ik} \\
    &= \sum_{i=1}^n \sum_{k=0}^K (\log \pi_k + \log f_k(X_i) + Y_i \log p_k(X_i) + (1 - Y_i) \log (1-p_k(X_i))) \tau_{ik} \\
\end{align}
$$

The \textbf{maximization} step now consists in deriving $Q(\widehat{\theta}, \theta)$ regarding each parameters in $\theta$
so that we obtain an either explicit value or iterative procedure to compute the next iterate of $\widehat{\theta}$.
\begin{itemize}
    \item Maximization regarding $\pi_k$ under constraint that $\sum_{k=0}^K \pi_k = 1$ can be solved explicitly using Lagrange duality:
    $$
    \pi_k^* = \frac{1}{n} \sum_{i=1}^n \tau_{ik}
    $$
    \item Maximization regarding $(\mu_k, sigma_k)$ is given by the maximum of likelihood estimator on $\sum_{i=1}^n \tau_{ik} \log f_{k}(X_i)$:
    $$
    \begin{align}
        \mu_k^* &= \frac{1}{\sum_{i=1}^n \tau_{ik}} \sum_{i=1}^n \tau_{ik} X_i \\
        \sigma_k^* &= \frac{1}{\sum_{i=1}^n \tau_{ik}} \sum_{i=1}^n \tau_{ik} (X_i - \mu_k^*)(X_i - \mu_k^*)^\top
    \end{align}
    $$
    \item Maximization regarding $(W_{e,k}, W_{x,k})$ is not explicit, and requires a fixed point algorithm like gradient descent to determine an estimate of the optimal parameters.
    The iterations using full batch gradient descent are given below, with learning rate $\alpha$:
    $$
    \begin{align}
        W_{e,k}^{l+1} &\leftarrow W_{e,k}^{l} - \alpha \sum_{i=1}^n (p_k(X_i) - Y_i) \tau_{ik} e_k \\
        W_{x,k}^{l+1} &\leftarrow  W_{x,k}^{l} - \alpha \sum_{i=1}^n (p_k(X_i) - Y_i) \tau_{ik} X_i
    \end{align}
    $$
    Each iteration should be confronted to the maximization criterion, so that each iterate improves $Q(\widehat{\theta}, \theta)$:
    $$
    Q(\widehat{\theta}^{l+1}, \theta) \geq Q(\widehat{\theta}^{l}, \theta)
    $$
    In the end, only the best improvement iterate is kept for $W_{e,k}^*$ and $W_{x,k}^*$. Note that other methods could be used like SGD or CMAES as implemented during the internship.
\end{itemize}

After the maximization, we can update $\widehat{\theta}$ with the previously computed parameters, and redo the (E) and (M) steps up to convergence.
The convergence can be measured relatively to $ Q(\widehat{\theta}, \theta)$, so that for a threshold $\epsilon$, we can use the following stopping criterion:
$$
\frac{|Q(\widehat{\theta}^{h+1}, \widehat{\theta}^{h}) - Q(\widehat{\theta}^{h}, \widehat{\theta}^{h})|}{|Q(\widehat{\theta}^{h}, \widehat{\theta}^{h})|} \leq \epsilon
$$

We now have a ready-to-go Gaussian Mixture Linear Classifier that we can benchmark on an suited dataset against other common methods.

\subsubsection{Dataset generation}

Before we benchmark the Gaussian Mixture Linear Classifier, we need to generate a dataset that is well suited to its usage.
Hence, we set random parameters for $\theta$ and generate a new dataset following the latent framework we have set previously:
\begin{itemize}
    \item For all $k \leq K$, generate $n/K$ points following a gaussian parameterized by $(\mu_k, \sigma_k I)$.
    $X$ is given by each point coordinate, $Z$ by the gaussian from which the point was generated.
    \item For each sample, characterized by ($X_i$, $Z_i$), draw the label of the sample as $Y_i \sim \mathcal{B}(\sigma(W_{e,Z_i}^T e_{Z_i} + W_{x,Z_i} ^ T X_i))$.
\end{itemize}

The following figure illustrates a generation with $K=2$:
\begin{figure}[H]
    \center
    \includegraphics[scale=0.7]{images/samples_gmm_lc}
    \caption{Generated samples out of $K=2$ gaussians, labelized using a logistic model (label 1 or 0).
    The density of the hidden gaussians is represented in the background using shades of grey (black intense, white almost 0)}
    \label{fig:samples_gmm_lc}
\end{figure}

Note on the previous figure \ref{fig:samples_gmm_lc} that the limit between the labels in each subset gaussian is not sharp,
as it is sorted out of a probabilistic modelisation (logistic).
Consequently, an interesting observation can be made as we force the gaussians to have small values of mean and variance.
Indeed, as shown on the next figure, if we take the same parameters as previously and divide them by a factor 10,
the logistic model is ill conditioned.
\begin{figure}[H]
    \center
    \includegraphics[scale=0.7]{images/samples_gmm_lc_badconditioning}
    \caption{Generated samples out of $K=2$ gaussians, using the same parameters as for figure \ref{fig:samples_gmm_lc} divided by a factor 10.}
    \label{fig:gmm_lc_badconditioning}
\end{figure}

All points are close to the frontier in terms of norm on figure \ref{fig:gmm_lc_badconditioning}.
Since we do not exploit any normalization term in the sigmoid modelization, this leads to a blurry area in which the linear separation model is not useful at all.
This situation is heavily problematic as it prevents us from performing a general benchmark on that dataset.
Indeed, the further larger the variance is, with sufficient samples, the better the linear approximation will be and therefore the better the modelization gets.
On the contrary, with smaller variance the linear model isn't descriptive of the generated samples as they are all close to the frontier, therefore with heavily noisy labelization.
\subsection{Dirichlet Mixture Linear Classifier}

\subsubsection{Framework and objectives}

In the previous section, we have derived a very classical model (Gaussian Mixture Model) into a linear classifier using the EM algorithm.
However, gaussian latent modelization is far from general, and may not be suited to our specific usage on the microbiota.
Indeed, after performing the previous method onto our microbiota dataset, it turned out to be performing just as bad as the classical logistic regression, no matter the chosen latent space dimension.
Therefore, we deduce that the gaussian latent modelization is not adapted to our practical settings. \\

As we analyze the data, we observe that each $X_i$ belongs to the simplex.
A natural distribution supported on the simplex is the Dirichlet distribution, parameterized by $\alpha = (\alpha_1, \dots, \alpha_p)$ where $p$ is the dimension of $X_i$.
We denote the dirichlet distribution by $\mathcal{D}(\alpha)$, for which the density is given by the following:
$$
f(x|\alpha) = \frac{\Gamma \left( \sum_{j=1}^p \alpha_j \right)}{\Pi_{j=1}^p \Gamma (\alpha_j)} \Pi_{j=1}^p x_j^{\alpha_j - 1}
$$
where $\Gamma$ denotes the gamma function.
For notation simplicity, we also introduce the digamma function that will play a key role in our model:
$$
\psi(\alpha) = \frac{d}{d \alpha} \log \Gamma(\alpha) = \frac{\Gamma '(\alpha)}{\Gamma(\alpha)}
$$

Since we are doing a Dirichlet mixture model, we have $K$ Dirichlet distributions to handle.
We introduce the notation $\alpha^{(k)}$ to parameterize the $k$-th Dirichlet distribution. \\

As previously, we first perform the \textbf{expectation} step obtain the same result with a different conditional a priori distribution on $X|Z$:
$$
\begin{align}
    p_{\widehat{\theta}}(Z_i = k | X_i, Y_i) &= \frac{\hat{\pi}_k f(X_i | \hat{\alpha}^{(k)}) (Y_i \hat{p}_k(X_i) + (1 - Y_i)(1 - \hat{p}_k(X_i)))}{\sum_{j=1}^K \hat{\pi}_j f(X_i | \hat{\alpha}^{(j)}) (Y_i \hat{p}_j(X_i) + (1 - Y_i)(1 - \hat{p}_j(X_i)))} \\
                                             &= \tau_{ik}
\end{align}
$$

We can now evaluate $Q(\widehat{\theta}, \theta)$ for any $\theta$, which enables us to perform the \textbf{maximization} step:
\begin{itemize}
    \item The maximization over $pi_k$ under the simplex constraint $\sum_{k=1}^k \pi_k = 1$ is again given by Lagrange duality as:
        $$
        \pi_k^* = \frac{1}{n} \sum_{i=1}^n \tau_{ik}
        $$
    \item The maximization over $\alpha^{(k)_j}$ is not straightforward on the other hand, and requires a fixed point algorithm.
          Indeed, deriving over $\alpha^{(k)_j}$ we obtain:
        $$
        \begin{align}
            \partial_{\alpha^{(k)}_j} Q(\widehat{\theta}, \theta) &= \sum_{i=1}^n \left(\psi \left(\sum_{l=0}^K \alpha_{l}^{(k)} \right) - \psi(\alpha_{j}^{(k)}) + \log x_{ij} \right) \tau_{ik} \\
                                                                  &= \left(\psi \left(\sum_{l=0}^K \alpha_{l}^{(k)} \right) - \psi(\alpha_{j}^{(k)})\right) \sum_{i=1}^n \tau_{ik} + \sum_{i=1}^n \tau_{ik} \log x_{ij}
        \end{align}
        $$
        Hence, as we look for $\partial_{\alpha^{(k)}_j} Q(\widehat{\theta}, \theta) = 0$, we obtain:
        $$
        \psi(\alpha_{j}^{(k)}) - \psi \left(\sum_{l=0}^K \alpha_{l}^{(k)} \right) = \frac{\sum_{i=1}^n \tau_{ik} \log x_{ij}}{\sum_{i=1}^n \tau_{ik}}
        $$
        Thankfully, \cite{dirichlet_digamma_trick} provides a few tricks to solve iteratively such equation, so that we can iterate as follows (5 steps are sufficient to obtain high-accuracy solution according to \cite{dirichlet_digamma_trick}):
        $$
        \begin{align}
            \alpha_{j}^{(k)} &\leftarrow \psi^{-1}\left(\frac{\sum_{i=1}^n \tau_{ik} \log x_{ij}}{\sum_{i=1}^n \tau_{ik}} + \psi \left(\sum_{l=0}^K \hat{\alpha}_l^{(k)} \right)  \right) \\
            \hat{\alpha}_j^{(k)} &\leftarrow \alpha_{j}^{(k)}
        \end{align}
        $$
        However, this solution is a lower bound to the true objective, which makes our EM a generalized version of it.
    \item The maximization over $(W_{e,k}, W_{x,k})$ is also given by a fixed point algorithm, which ends up being the same computation as previously for the Gaussian case:
        $$
        \begin{align}
            W_{e,k}^{l+1} &\leftarrow W_{e,k}^{l} - \alpha \sum_{i=1}^n (p_k(X_i) - Y_i) \tau_{ik} e_k \\
            W_{x,k}^{l+1} &\leftarrow  W_{x,k}^{l} - \alpha \sum_{i=1}^n (p_k(X_i) - Y_i) \tau_{ik} X_i
        \end{align}
        $$
\end{itemize}

\subsubsection{Dataset generation}

Now that we have defined the EM algorithm in the previous section, we aim at generating a dataset to benchmark the dirichlet mixture classifier.
Following a similar procedure as for the gaussian case, we are able to generate a dataset that matches a dirichlet mixture and is labeled following the sigmoid modelisation.
The next figure illustrates a given generation:
\begin{figure}[H]
    \center
    \includegraphics[scale=0.7]{images/samples_dirichlet_lc}
    \caption{Generated samples out of $K=2$ dirichlet distributions, labelized using the sigmoid modelisation on $\mathbb{P}(Y_i = 1)$.}
    \label{fig:dirichlet_lc_badconditioning}
\end{figure}

As previously, since the data lives in the simplex, they are too close to the border of the next label set, which ends up creating a blurry dataset for which the
linear model is not relevant anymore.

% Discrete VAE
\newpage
\section{Variational methods}

In this section, we aim at resourcing some variational methods that we are going to use in this study.
\section{Variational Auto-Encoders}

\subsection{Framework and optimization objective}

We are interested in another kind of latent models, this time based on variational inference
results to achieve a new kind of deep latent structure: the Variational Auto-Encoder (VAE).
These latent models were introduced in 2013 by Kingma, better described in a more in depth paper in 2019: \cite{kingma2019introduction}

Once again, we assume the observations $X$ to be modelizable by a given distribution
parameterized by $\theta$:

$$
X \sim p_{\theta}(x)
$$

Determining $\theta$ holds to find one $\theta^*$ that would optimize a given objective, generally chosen as the maximum of likelihood. Indeed, if $\theta^* \in arg\max_{\theta} p_{\theta}(x)$, then such $\theta^*$ maximizes the density around the dense areas of the observations, which makes them highly likely to happen under such distribution $p_{\theta^*}$. Hence, the maximum likelihood is a natural criterion:
$$
\theta^* \in arg\max_{\theta} p_{\theta}(x)
$$

However, such modelization does not include a latent structure. As a result, we try to enforce it by rewriting the objective as follows:

$$
p_{\theta}(x) = \int_{\mathcal{Z}} p(x,z) dz 
$$

Using Bayes decomposition, we obtain a computable objective:
$$
p_{\theta}(x,z) = p_{\theta}(z) p_{\theta}(x|z)
$$

Recall that the prior $p_{\theta}(z)$ and the a priori $p_{\theta}(x|z)$ are defined by the framework (ex: Bernoulli prior and Gaussian posterior gives the Gaussian mixture framework). However, the computation of the evidence $p_{\theta}(x)$ is generally intractable in practice, which also leads to a non-tractable posterior distribution: $p_{\theta}(z|x)$. As a result, not being able to compute the evidence leads to not being able to provide a gradient regarding $\theta$, so we can not perform the backpropagation in a deep learning approach.

Note that there exist approximate inference techniques to compute the evidence and the posterior, but these are quite expensive and often yield poor convergence results.

To overcome this issue, we introduce a smart rewriting of the objective using variational inference. Indeed, let $q_{\Phi}(z|x) \approx p_{\theta}(z|x)$ to be learnt over $\Phi$, one can write:
$$
\begin{align}
    \log p_{\theta}(x) &= \mathbb{E}_{q_{\Phi}(z|x)}[\log p_{\theta}(x)] \\
    &= \mathbb{E}_{q_{\Phi}(z|x)}\left[\log \frac{p_{\theta}(x)}{q_{\Phi}(z|x)} \frac{q_{\Phi}(z|x)}{p_{\theta}(z|x)}\right]
    \\
    &= \underbrace{\mathbb{E}_{q_{\Phi}(z|x)} \left[\log \frac{p_{\theta}(x)}{q_{\Phi}(z|x)}\right]}_{ELBO(q_\phi(z|x), p_{\theta}(x,z))} + D_{KL}\left[q_{\Phi}(z|x) \Vert p_{\theta}(z|x)\right]
\end{align}
$$

The first term of that decomposition is generally called the Evidence Lower BOund (ELBO), as it marks a lower bound to the evidence $\log p_{\theta}(x)$ since the KL divergence is a positive quantity:
$$
\log p_{\theta}(x) \ge ELBO(q_\phi(z|x), p_{\theta}(x,z))
$$

$q_{\Phi}(z|x)$ is an approximation of the true posterior $p_{\theta}(z|x)$ that we aim at learning in a family of distributions. For instance, 
$$
q_{\Phi}(.|x) \sim \mathcal{N}(\mu(x), \Sigma(x))
$$ 
would be an approximation of the true posterior by a Gaussian distribution. Notice that the true posterior may very not likely be Gaussian, which creates a first complexity error in our model.

Despite being a lower bound on the true maximum likelihood objective, the ELBO is actually tractable. Indeed, as we continue the computation:
$$
\begin{align}
    ELBO(q_\phi(z|x), p_{\theta}(x,z)) &= \mathbb{E}_{q_{\Phi}(z|x)}[\log p_{\theta}(x,z)] - \mathbb{E}_{q_{\Phi}(z|x)}[\log q_{\Phi}(z|x)] \\
    &= 
\mathbb{E}_{q_{\Phi}(z|x)}[\log p_{\theta}(x|z)] - D_{KL}[\log q_{\Phi}(z|x) \Vert p_{\theta}(z)] \\\end{align}
$$

Another remarkable fact, is that when maximizing the ELBO, we are actually minimizing the KL divergence between the estimated and the true posterior. Hence, one can define the ELBO as a suboptimal objective to our problem that we get to maximize to obtain $(\Phi^*, \theta^*)$, the parameters of our model.

\subsection{Reparameterization trick}

Even though the gradient of the ELBO is well defined for $\theta$, it is not possible to compute it relatively to $\Phi$ it requires samples from the approximation to the posterior $q_{\Phi}(z|x)$ to compute $\mathbb{E}_{q_{\Phi}(z|x)}[\log p_{\theta}(x|z)]$. 

Since, sampling is not a differentiable operation, we make use of the change of variable formula, so that for a bijective transformation $z = \phi_x(\epsilon)$, we get:
$$
p(z) = p(\epsilon) \det \left| \frac{\partial \epsilon}{\partial z}\right|
$$

Hence, if we take $\epsilon$ a random variable of density $p(\epsilon)$ that does not depend on $\theta$, $\Phi$ nor $x$, so that $z = \phi_x(\epsilon)$, for any $L_1$ function $f$,
$$
\mathbb{E}_{q_{\Phi}(z|x)}[f(z)] = \mathbb{E}_{p(\epsilon)}[f(z)]
$$

As a result, the samples are not obtained through $q_{\Phi}$ anymore but through $p(\epsilon)$, so that can safely perform derivation of the ELBO relatively to $\Phi$ and backpropagate our gradient through the network.

\subsection{Architecture}

The vanilla architecture of the VAE is described by the following illustration:

[ILLUSTRATION VAE]

The first part is generally called the encoder, as it turns a sample $x$ into its latent representation $z$ by modelizing the posterior $q_{\Phi}(z|x)$. The second part is then called the decoder, as it throws a latent representation in the sample space. The latest can even serve as a generative architecture, as one can sample from the latent space through $q_{\Phi}(z|x)$, and decode it to obtain a new sample.

As we can see more clearly in that illustration, we can see that $\Phi$ and $\theta$ are trained jointly through the ELBO, both serving for one part of the VAE at a time.

The training procedure is straightforward: the entry is a sample $x$ and the output objective is the same sample $x$. We aim at train the VAE for learning the data space and its latent representation by learning how to reconstruct the samples through it.




\newpage
\section{VQ-VAE}

\subsection{Quick overview}

Introduced in \cite{vq_vae_paper}, VQ-VAE architecture provides a framework to compute
discrete posterior distributions $q_{\Phi}(z|x)$.
To compare that model with the VAE, we start by introducing the architecture of the model for which an illustration is given below:

\begin{figure}[H]
    \centering
    \includegraphics[width=.9\textwidth]{images/vq_vae_architecture}
    \caption{VQ-VAE architecture (source: Medium)}
    \label{fig:vq_vae_architecture}
\end{figure}

As we can see on figure \ref{fig:vq_vae_architecture}, the major difference with the vanilla VAE architecture
lies in the vector quantization step which enables to project the output of the encoder denoted by $z_e(x)$
onto a discrete embedding dictionary $(e_1, \dots, e_K)$ by a simple distance argument:
$$
k = arg\min_{j} \Vert z_q(x) - e_j \Vert_2, \quad z_q(x) = e_k
$$

The projection of $z_e(x)$ on that discrete dictionary is denoted by $z_q(x)$, and serves as the input of the decoder.
For further visual representation of the vector quantization layer, an illustration is given below.

\begin{figure}[H]
    \centering
    \includegraphics[width=.7\textwidth]{images/vq_vae_quantization}
    \caption{Architecture of the VQ-VAE: quantization layer (source: Medium)}
    \label{fig:vq_vae_quantization}
\end{figure}

Looking at the previous figures, we can grasp the challenge of backpropagation in such model with discrete prior and posterior.
In the next section, we enter in the mathematical definition of the objective and how to train this architecture.

\subsection{Framework and optimization objective}

Contrary to the vanilla VAE, we have the following categorical distributions assumption:
\begin{itemize}
    \item The prior $p_{\theta}(z)$ is categorical.
          In the original paper, it is taken as uniform supported in $\{1, \dots, K\}$ during the training.
          When the training is over, it is fit to an autoregressive distribution through a PixelCNN (see \cite{pixel_cnn_paper}).
          It is left as an exploration research field to be able to learn the prior while training the model.
    \item The posterior $q_{\Phi}(z|x)$ is categorical and set to the following:
          $$
          q_{\Phi}(k|x) = \mathds{1}_{\{k = arg\min_{j} \Vert z_q(x) - e_j \Vert_2\}}
          $$
\end{itemize}

This modelization of the posterior enables to obtain a discrete latent space, but it does not allow to differentiate regarding $\Phi$.
As a result, the authors suggest two possible strategies:
\begin{itemize}
    \item \textit{Straight-through}: propagate the gradient through the discrete part (vector quantization layer) without changing it.
          The intuition is that the gradient propagated from the encoder contains sufficient information to update the encoder accordingly, but this is just intuition.
    \item \textit{Subgradient}: compute the subgradient of the quantization layer (unexplored yet)
\end{itemize}

The optimization objective of the VQ-VAE is based on the ELBO, that one can compute as follow:
$$
\begin{align}
    ELBO(q_{\Phi}(z|x), p_{\theta}(z|x)) &= \mathbb{E}_{q_{\Phi}(z|x)}[\log p_{\theta}(x,z)] - \mathbb{E}_{q_{\Phi}(z|x)}[\log q_{\Phi}(z|x)] \\
    &= \mathbb{E}_{q_{\Phi}(z|x)}[\log p_{\theta}(x|z)] - D_{KL}[q_{\Phi}(z|X) \Vert p_{\theta}(z)]
\end{align}
$$

Notice then that:
\begin{itemize}
    \item Since $q_{\Phi}(k|x) = \mathds{1}_{\{k = arg\min_{j} \Vert z_q(x) - e_j \Vert_2\}}$, we have:
          $$
          \begin{align}
              \mathbb{E}_{q_{\Phi}(z|x)}[\log p_{\theta}(x|z)] = \log p_{\theta}(x|z_q(x))
          \end{align}
          $$
    \item Since $Z \sim \mathbb{U}(\{1, ..., K\})$, $\mathbb{P}(Z=k) = \frac{1}{K}$.
          Also, notice that $q_{\Phi}(z_q(x)|x) = 1$ by definition of $q_{\Phi}(z|x)$ and $z_q(x)$.
          Combining those results, we obtain:
          $$
          \begin{align}
              D_{KL}[q_{\Phi}(z|x) \Vert p_{\theta}(z)] &= \mathbb{E}_{q_{\Phi}(z|x)}\left[\log \frac{q_{\Phi}(z|x)}{p_{\theta}(z)} \right] \\
              &= \log \frac{q_{\Phi}(z_q(x)|x)}{p_{\theta}(z_q(x))} \\
              &= \log K
          \end{align}
          $$
          Therefore, this KL divergence does not impact the optimization objective as it does not depend on ($\theta$, $\Phi$).
\end{itemize}

Computing the previous quantities, we would obtain the following suboptimal objective for the VQ-VAE:
$$
ELBO(\Phi, \theta) = log p_{\theta}(x|z_q(x))
$$

However, such objective does not enable to learn the dictionary since we use a straight-through approach over the quantization layer.
To update the dictionary, the authors suggest to add the following term to the loss:
$$
\Vert sg[z_e(x)] - e \Vert_2^2
$$
Where $sg$ denotes the stop-gradient operator, meaning we do not consider any gradient after the given operation.
\medskip

Finally, to prevent embedding space over expansion, the authors suggest the addition of a commitment loss parameterized by $\beta > 0$:
$$
\beta \Vert z_e(x) - sg[e] \Vert_2^2
$$
Indeed, the embedding space was unconstrained so far, so its dimension could grow arbitrarily.
Intuitively, adding that term forces the model to commit to a given embedding.
\medskip

Overall, we obtain this final objective to maximize:
$$
\mathcal{L}(\theta, \Phi) = \log p_{\theta}(x|z_q(x)) + \Vert sg[z_e(x)] - e \Vert_2^2 + \beta \Vert z_e(x) - sg[e] \Vert_2^2
$$

\subsection{Discussion over some limiting aspects}

Even though the previous objective seems natural, we only rigorously justified the first term thanks to the ELBO.
Indeed, the dictionary update loss as well as the commitment loss keep dropping out from nowhere, while they seem necessary to ensure the training of our model.
\medskip

Furthermore, we did not really deal with the non differentiability of the vector quantization loss, and the straight-through estimator
can definitely be criticized about what information it actually provides to the encoder to update appropriately.
\medskip

Finally, the usage of the PixelCNN after the training seems very unnatural, and could significantly boost the model as the PixelCNN did show amazing performances so far.
\newpage
\subsection{PixelCNN}

\subsubsection{Overview}

Introducing the VQ-VAE, we have seen that an under-table tool that was being used is the Pixel CNN,
first introduced in \cite{pixel_cnn_paper}.
In the context of the VQ-VAE, it plays a major role after training by learning the prior $p(z)$, that was set
to a discrete uniform previously.
This makes a drastic difference with the VAE, as we are not setting the prior ourselves as it's being learnt and modeled by the PixelCNN in this process.
\medskip

Indeed, the PixelRNN/PixelCNN architectures are sequential deep neural networks that aims at modeling the distribution of a data space
in an autoregressive fashion.
Their main characteristic is that they take advantage of the structure of an image to learn the data space distribution.
\begin{itemize}
    \item \textit{PixelRNN}: Bi-directional recurrent networks with variable smart directions are used to model the spatial dependencies between pixels.
    (3 architectures are presented in the original paper, but our focus will be on the PixelCNN here).

    \item \textit{PixelCNN}: the dependencies between the pixels is modeled through stacking of masked convolution layers (it's faster since the receptive field is bounded by the size of the convolution), no pooling layer is used.
    The masking ensures that we keep an autoregressive estimation of a new pixel, without seeing the future pixels.
\end{itemize}

The following figure illustrates the masked convolution technique.
\begin{figure}[H]
    \center
    \includegraphics[scale=0.9]{images/masked_convolution_pixel_cnn}
    \caption{Masked convolution on an image (largest square), the big square represents the receptive field of the black pixel, the white area is masked}
    \label{fig:masked_convolution_pixel_cnn}
\end{figure}

Hence, given $x$ an image represented by its pixels $x = (x_1, \dots, x_n)$, as usual for these distribution modeling problems, we aim at finding $\theta^* = arg\max_{\theta} p_{\theta}(x)$ in a family of distributions parameterized by $\theta$: the maximum of likelihood.
Contrary to the usual independent framework, we consider a local dependency between pixels given by the receptive field of our convolution.
This local dependency is limited to the already seen pixels only as well, thanks to the masked convolution.
\medskip

Furthermore, rather than using continuous outputs, these architecture use a softmax layer to determine the pixel of a given generation,
leading to a discrete prior rather than a continuous one, which is required for the VQ-VAE for instance.
As a result, the distribution of a pixel conditionally to the ones in its receptive field is given by a multinomial in $\{0, \dots, 256\}$.
\medskip

\subsubsection{Usage in the VQ-VAE}

Once we have trained the VQ-VAE, the original paper states that we can replace the uniform prior on $p(z)$ by a PixelCNN to model the prior.
To perform the training of the PixelCNN, we turn the samples $x$ in their latent representations $z$, and train the PixelCNN over the latent representations $z$.
This way, we have created dependency between the $z$ in the latent feature mapping, and we obtain a prior over their distribution modeled by the PixelCNN.



% Microbiota analysis
\newpage
\section{Microbiota analysis}

This section aims at using the previous elements to design microbiota adapted latent methods.
\subsection{Microbiota dataset}

As we look into the microbiota data, we notice a major phylogenetic architecture to describe various levels of precision in the microbiota composition.
Indeed, such phylogenetic structure can be represented as a tree as on the following figure:
\begin{figure}[H]
    \center
    \includegraphics[scale=1]{images/abundance_tree_phylogenetic}
    \caption{Phylogenetic tree example with abundance data (in the nodes) at each layer of the tree.
    Each node represents a bacterium specie at a given precision layer in the tree. From \cite{microbiome_deeplearning_research}.}
    \label{fig:phylogenetic_tree}
\end{figure}

Such structure can not be used directly in a machine learning system since it's not a vectorizable representation.
Hence, we first suggest to transform the tree in a matrix to image structure as in the following example:
\begin{figure}[H]
    \center
    \includegraphics[scale=1]{images/tree_to_image}
    \caption{Phylogenetic tree to image representation:
    opacity of the pixel relates to the abundance of the species at the given level of precision (normalized between 0 and 1).
    From \cite{microbiome_deeplearning_research}.}
    \label{fig:phylogenetic_tree_to_img}
\end{figure}

To give some notations, we introduce the following framework:
\begin{itemize}
    \item The maximum depth of a phylogenetic tree is $D$, the maximum number of nodes $N_T$, and the maximum amount of unique species at any level is $U$.
    \item $X_i$ is the abundance matrix of the individual $i$ (image of size $D \times U$) , which is observed.
    \item $T_i$ is the adjacent matrix of the phylogenetic tree of individual $i$, of size $N_T \times N_T$, which is observed.
    We could use any other encoding of a tree (contour function, depth function, $\dots$).
    \item We assume that we have $n$ individuals observed through $(X_i, T_i)_{1 \leq i \leq n}$ i.i.d samples.
\end{itemize}

\subsection{Simple generative model: no latent variable}

\subsubsection{Context and objective}

In first approximation, we would like to define a generative model that does not exploit any latent structure.
Such model, parameterized by $\theta$, aims at finding an optimal distribution in the sense of the maximum of likelihood,
within a family of distributions yet to be defined.
The maximum likelihood objective is given below as:
$$
\theta^* = arg\max_{\theta} p_{\theta}(X, T)
$$
One can rewrite the joint distribution as follow:
$$
\begin{align}
    p_{\theta}(X, T) &= \prod_{i=1}^n p_{\theta}(X_i, T_i) \\
                    &= \prod_{i=1}^n p_{\theta}(T_i) p_{\theta}(X_i | T_i)
\end{align}
$$

As a result, to compute this objective, we need to define a prior $p_{\theta}(T_i)$ that generates trees,
and a posterior distribution $p_{\theta}(X_i | T_i)$ that generates abundance data from a sampled tree.

\subsubsection{Design of the prior}

We aim at defining a parameterized distribution $p_{\theta}(T)$ from which one can sample trees.
We would also like this distribution to model the trees of the microbiota dataset, meaning that
the generated trees should look like the ones from the dataset as well, and respect the phylogenetic constraints. \\

Consequently, we introduce this first simple generative process to characterize $p_{\theta}(T)$:
\begin{algorithm}[H]
    \caption{Independent Branches Tree (IBT) sampling}
    \begin{algorithmic}
        \STATE Define the tree $T$
        \STATE Sample $K \sim \beta$ the number of leaf entities in the tree
        \For{$i \in \{1, \dots, K\}$}
            \STATE \quad Define the branch $b_i$
            \STATE \quad Set the root of the tree to the lowest precision level entity as $u_0^{(i)}$
            \STATE \quad Define $j = 0$
            \While{$u_{j}^{(i)}$ is not the empty node index}
                \STATE \quad \quad Append the node $u_{j}^{(i)}$ to the branch $b_i$
                \STATE \quad \quad Gather the possible children of the node $u_{j}^{(i)}$ as $V_{j}^{(i)}$
                \STATE \quad \quad Append the empty node index in $V_{j}^{(i)}$
                \STATE \quad \quad Sample $u_{j+1}^{(i)}$ following $g_{\pi_{u_{j}^{(i)}}}$ a discrete distribution over $V_{j}^{(i)}$ with weights $\pi_{u_{j}^{(i)}}$
                \STATE \quad \quad $j \leftarrow j + 1$
            \EndWhile
            \STATE \quad Add the branch $b_i$ to the tree $T$.
        \EndFor
        \RETURN{$T$}
    \end{algorithmic}
    \label{alg:algorithm}
\end{algorithm}

The IBT generates trees with independent branches, so that each node in a branch is dependent from the previous ones,
while each branch starts from the same root and are i.i.d.
Hence, such structure is one the most simple one could think of when generating trees. \\
Another step would be to add dependency between branches, for instance looking at each generation level what other entities
have been selected in all the branches at the same level before.
Though that would require a supplementary modelization step that we leave on the side for now. \\

Following the IBT modelization, we can compute the prior as:
$$
\begin{align}
    p_{\theta}(T) &= p_{\theta}(K = |T|, b_1, \dots, b_{K}) \\
                &= p_{\beta}(K=|T|) \prod_{i=1}^{|T|} p_{\theta}(b_i) \\
                &= \beta_{|T|} \prod_{i=1}^{|T|} p_{\theta}(u_1^{(i)}, \dots, u_{|b_i|}^{(i)}) \\
                &= \beta_{|T|} \prod_{i=1}^{|T|} \underbrace{p(u_1^{(i)})}_{1} p_{\pi_{u_1^{(i)}}}(u_2^{(i)} | u_1^{(i)}) \dots p_{\pi_{u_{|b_i|-1}^{(i)}}}(u_{|b_i|}^{(i)} | u_{|b_i|-1}^{(i)}) \\
                &= \beta_{|T|} \prod_{i=1}^{|T|} \prod_{j=1}^{|b_i|-1} p_{\pi_{u_j^{(i)}}}(u_{j+1}^{(i)} | u_{j}^{(i)}) \\
                &= \beta_{|T|} \prod_{i=1}^{|T|} \prod_{j=1}^{|b_i|-1} \pi_{u_j^{(i)} \rightarrow u_{j+1}^{(i)}} \right
\end{align}
$$

\subsubsection{Design of the posterior}

Now that we have a way to generate trees, we need to define an explicit stochastic relationship between the abundance
data and the structure of the tree.
Such inevitably exists, as when observing $T$, the abundance of an entity that isn't present in $T$ is necessarily 0.
Similarly, it is highly likely that when observing the presence of certain entities in $T$ that induces a high abundance of
another neighbour entity (interaction between bacteria). \\

For context, we recall that $X$ is a matrix of shape $(D, U)$.
We denote by $X^{(l)}$ the $l$-th line of the abundance matrix, for which up to $U_l$ elements should be non-zero. \\

Since we would like an explicit model here, we design a posterior $p_{\theta}(X|T)$ which is markovian relatively to the layers of the tree:
$$
p_{\theta}(X^{(l)} | X^{(1:l-1)}, T) = p_{\theta}(X^{(l)} | X^{(l-1)}, T)
$$
We assume the following framework:
\begin{itemize}
    \item We denote $X^{(l)} = (x_l^{(1)}, \dots, x_{l}^{(U)})$ the abundance vector at layer $l$.
    \item $X^{(1)} = [1, 0, \dots, 0]$, since it's the root of the tree, only one entity gets the whole weight.
    \item We assume that for all $l \geq 2, X^{(l)} | T \sim \mathcal{D}(\alpha_l)$.
    \item Each value in $X^{(l)}$ is restricted by the following set of constraints:
        \begin{itemize}
              \item If node $k$ at layer $l-1$ has no child, then we create a ghost child for which the abundance value is the same as the parent.
                    This node should be removed at the end of the procedure.
                    This is an artificial way of keeping the Dirichlet modelisation, so that all rows sum to 1 while generating X.
              \item If node $k$ at layer $l-1$ has one child, then its abundance value is the same for the child node.
              \item If node $k$ at layer $l-1$ has at least two children, the children abundance sums to the parent's abundance value.
        \end{itemize}
        We denote by $\mathcal{C}_T$ the set of vectors that verify the previous constraints for the tree $T$.
        Hence, we obtain the following conditional distribution:
        $$
        \begin{align}
            p_{\gamma_l}(X^{(l+1)} | X^{(l)}, T) &= \frac{p_{\gamma'_l}(X^{(l+1)}, X^{(l)}, T)}{p_{\alpha_l}(X^{(l)}, T)} \\
                                            &= \frac{p_{\alpha_{l+1}}(X^{(l+1)} | T)}{p_{\alpha_{l}}(X^{(l)} | T)} p(X^{(l)} | X^{(l+1)}, T) \\
                                            &= \frac{p_{\alpha_{l+1}}(X^{(l+1)} | T)}{p_{\alpha_{l}}(X^{(l)} | T)} \mathds{1}_{\mathcal{C}_T}\left(X^{(l)}, X^{(l+1)}\right)
        \end{align}
        $$
\end{itemize}

The whole posterior is then given by:
$$
\begin{align}
    p_{\theta}(X | T) &= \prod_{i=1}^n p_{\theta}(X_i | T_i) \\
                    &= \prod_{i=1}^n p_{\theta}(X_i^{(1)}, \dots, X_i^{(D)} | T_i) \\
                    &= \prod_{i=1}^n p(X_i^{(1)} | T_i) \prod_{l=1}^{D-1} p_{\gamma_j}(X_i^{(l+1)} | X_i^{(l)}, T_i) \\
                    &= \prod_{i=1}^n \mathds{1}_{X_i^{(1)} = e_1} \prod_{l=1}^{D-1} \frac{p_{\alpha_{l+1}}(X_i^{(l+1)} | T_i)}{p_{\alpha_{l}}(X_i^{(l)} | T_i)} \mathds{1}_{\mathcal{C}_{T_i}}\left(X_i^{(l)}, X_i^{(l+1)}\right) \\
                    &= \prod_{i=1}^n p_{\alpha_{D}}(X_i^{(D)} | T_i) \mathds{1}_{X_i^{(1)} = e_1} \prod_{l=1}^{D-1} \mathds{1}_{C_{T_i}}(X_i^{(l)} X_i^{(l+1)})
\end{align}
$$

Hence, to generate an abundance matrix $X$ out of a tree $T$, we just need to sample the highest precision layer of the tree
and roll it up to the top thanks to the tree structure that we observe.
As a result, this posterior is only characterized by one parameter $\alpha_D$ that is the dirichlet parameter of the ultimate layer of the tree. Sampling from the last layer $D$ has to be made explicit though.
We assume that the abundance at the last layer only depends on the last layer of the tree, so that:
$$
p_{\alpha_D}(X_i^{(D)} | T_i) = p_{\alpha_D}(X_i^{(D)} | T_i^{(D)})
$$
\newcommand{\nodeindexset}{\mathcal{L}^{(D)}_{T_i}}
Knowing the last layer structure $T_i^{(D)}$ enables us to tell for sure where some $0$ abundances are going to be.
Let's denote by $\nodeindexset$ the set of indexes so that the node associated to that index belongs to any branch at the level set $T_i^{(D)}$ in $T_i$:
$$
\nodeindexset = \left\{k \in \{1, \dots, U\} | u_k^{(D)} \in T_i^{(D)}\right\}
$$
We can now describe the last layer's abundance conditionally to the tree:
$$
\begin{align}
    p(X_i^{(D)} | T_i) &= p(X_i^{(D)} | T_i^{(D)}) \\
                                &= p\left(\left[X_i^{(D)}\right]_1, \dots, \left[X_i^{(D)}\right]_U | \bigcap_{k' \notin \nodeindexset} \left[X_i^{(D)}\right]_{k'} = 0\right) \\
                                &= \frac{p_{\alpha_D}\left( \bigcap_{k \in \nodeindexset} \left[X_i^{(D)}\right]_{k}, \bigcap_{k' \notin \nodeindexset} \left[X_i^{(D)}\right]_{k'} = 0 \right)}{\int p_{\alpha_D}\left( \bigcap_{k \in \nodeindexset} \left[\tilde{X}_i^{(D)}\right]_{k}, \bigcap_{k' \notin \nodeindexset} \left[X_i^{(D)}\right]_{k'} = 0 \right) d\tilde{X}} \\
                                &= \frac{\frac{1}{B(\alpha_D)} \prod_{k \in \nodeindexset} \left[X_i^{(D)}\right]_{k}^{\alpha_D^{(k)} - 1} \prod_{k' \notin \nodeindexset} \left[X_i^{(D)}\right]_{k'}^{\alpha_D^{(k')} - 1}}{\frac{1}{B(\alpha_D)} \prod_{k' \notin \nodeindexset} \left[X_i^{(D)}\right]_{k'}^{\alpha_D^{(k')} - 1} \prod_{k \in \nodeindexset} \int_{0}^{1} \left[\tilde{X}_i^{(D)}\right]_k^{\alpha_D^{(k)} - 1} d\left[\tilde{X}_i^{(D)}\right]_k} \\
                                &= \prod_{k \in \nodeindexset} \alpha_D^{(k)} \left[X_i^{(D)}\right]_k^{\alpha_D^{(k)} - 1}
\end{align}
$$
Plugin that result into the whole distribution, we obtain that final formulation of the posterior distribution:
$$
    p_{\theta}(X | T) = \prod_{i=1}^n \prod_{k \in \nodeindexset} \alpha_D^{(k)} \left[\tilde{X}_i^{(D)}\right]_k^{\alpha_D^{(k)} - 1} \mathds{1}_{X_i^{(1)} = e_1} \prod_{l=1}^{D-1} \mathds{1}_{C_{T_i}}(X_i^{(l)} X_i^{(l+1)})
$$

\subsubsection{Optimization of the objective}

Recall the optimization objective, written under the maximum of the log-likelihood:
$$
\begin{align}
    \theta^* &= arg \max_{\theta} \log p_{\theta}(X,T) \\
            &= arg \max_{\theta} \sum_{i=1}^n \log p_{\theta}(X_i, T_i) \\
            &= arg \max_{\theta} \sum_{i=1}^n \log p_{\theta}(T_i) + \log p_{\theta}(X_i | T_i)
\end{align}
$$

Notice that the first term of the loss is made only of the prior, while the second one is made only of the posterior.
Since these two distributions do not share common parameters in $\theta$, we can proceed to optimize the parameters of each distribution
without taking care of the other term. \\

\newcommand{\transitionproba}{\pi_{a \rightarrow b}}
\newcommand{\transitionprobasum}{\pi_{u_j^{(k)} \rightarrow u_{j+1}^{(k)}}}
\newcommand{\transitionbranch}{b_{a \rightarrow b}^{(i)}}
\newcommand{\children}{\mathcal{C}}
\newcommand{\lagrangian}{\mathcal{L}}

Starting with the prior term, it is defined by the discrete distribution $\beta$ and the transition probabilities denoted as $(\transitionproba)_{(a,b) \in T}$
where $(a,b) \in T$ denotes a parent-to-child path. \\

The maximum likelihood objective for $\beta^*$ is given by:
$$
\begin{equation}
    \begin{aligned}
        \beta^*_k = arg \max_{\beta_k} \quad & \sum_{i=1}^n \log \beta_{|T_i|} \\
        \textrm{s.t.} \quad & \forall i, \beta \geq 0 \\
                      \quad & \sum_{i=1}^D \beta_i = 1
    \end{aligned}
    \label{eq:prior_beta_objective}
\end{equation}
$$

The previous optimization problem is rapidly solved through Lagrange duality, so that we obtain a straight-forward estimator:
$$
\fbox{
    \displaystyle \beta_k^* = \frac{1}{n} \sum_{i=1}^n \mathds{1}_{|T_i| = k}
}
$$

To compute $\transitionproba^*$, we introduce the set of branches that contain the path $a \rightarrow b$ for a given tree $T_i$:
$$
\transitionbranch = \{b \in T_i | (a,b) \in b\}
$$
For a given node $a$ we also denote by $\children(a)$ the set of possible children from $a$.
Note that $\children(a)$ is never empty, as every node can lead to stopping the branch, which means it contains the ghost empty node.

The optimization objective for $\transitionproba$ is given by:
$$
\begin{equation}
    \begin{aligned}
        \transitionproba^* = arg \max_{\transitionproba} \quad & \sum_{i=1}^n \sum_{k=1}^{|T_i|} \sum_{j=1}^{|b_i| - 1} \log \transitionprobasum \\
        \textrm{s.t.} \quad & \sum_{l \in \children(a)} \pi_{a \rightarrow l} = 1\\
        & \transitionproba \geq 0    \\
    \end{aligned}
    \label{eq:prior_transition_objective}
\end{equation}
$$

Such optimization objective is convex, with convex constraints that are solvable through Lagrange duality.
Computing the lagrangian we obtain:
$$
\lagrangian(\transitionproba, \lambda) = \sum_{i=1}^n \sum_{k=1}^{|T_i|} \sum_{j=1}^{|b_i| - 1} \log \transitionprobasum - \lambda_1 \transitionproba + \lambda_2 \left(\sum_{l \in \children(a)} \pi_{a \rightarrow l} - 1\right) + \iota_{\mathbb{R}^{+}}(\lambda_1, \lambda_2)
$$

The KKT conditions are then written as:
$$
\begin{align}
    0 &\in \partial_{\transitionproba} \lagrangian(\transitionproba, \lambda) \\
    \lambda_2 \left(\sum_{l \in \children(a)} \pi_{a \rightarrow l} - 1\right) &= 0 \\
    \lambda_1 &\geq 0 \\
    -\lambda_1 \transitionproba &\leq 0 \\
    -\lambda_1 \transitionproba &= 0 \\
\end{align}
$$

Using the first equality, we obtain:
$$
\transitionproba = \frac{1}{\lambda_1 - \lambda_2} \sum_{i=1}^{n} |\transitionbranch|
$$

Using the sum over all $l \in \children(a)$ on the previous result, assuming $\lambda_2 = 0$, the second KKT identity gives us:
$$
\sum_{l \in \children(a)} \pi_{a \rightarrow l} = \sum_{l \in \children(a)} \frac{\sum_{i=1}^n |\transitionbranch|}{\lambda_1} = 1
$$

Hence, we obtain that $\lambda_1 = \sum_{l \in \children(a)} \sum_{i=1}^n |\transitionbranch| = n_a$, the number of branches that goes through $a$.
Finally, we obtain the optimal transition probability:
$$
\fbox{
    \displaystyle\transitionproba^* = \frac{1}{n_a} \sum_{i=1}^n |\transitionbranch|
}
$$
One can verify that the KKT conditions are then satisfied as well. \\

Now, we compute the optimal parameter for the posterior that is given by $\alpha_D^* = (\alpha_D^{(1)}, \dots, \alpha_D^{(U)})$.
The optimization objective for each component is the following:
$$
(\alpha_D^{(j)})^* = arg \max_{\alpha_D^{(k)}} \sum_{i=1}^n \log p_{\alpha_D}(X_i^{(D)}|T_i) + \log \mathds{1}_{X_i^{(1)} = e_1} + \sum_{k=1}^{D-1} \log \mathds{1}_{C_{T_i}}(X_i^{(l)}, X_i^{(l+1)})
$$

Since the indicator function do not depend on the parameter, we only have to solve the following problem:
$$
(\alpha_D^{(j)})^* = arg \max_{\alpha_D^{(k)}} \sum_{i=1}^n \sum_{k \in \nodeindexset} \log \alpha_D^{(k)} + \left(\alpha_D^{(k)} - 1\right) \log \left[X_i^{(D)} \right]_k
$$
Deriving with respect to $\alpha_D^{(j)}$, if any tree has the node $j$, we get:
$$
\fbox{

    (\alpha_D^{(j)})^* = \left(\sum_{i=1}^n \mathds{1}_{j \in \nodeindexset} \log \frac{1}{\left[X_i^{(D)} \right]_j}\right)^{-1}

}
$$

%Since the indicator functions do not depend on the parameter, we end up on a known problem of maximum of likelihood for Dirichlet distribution \cite{dirichlet_digamma_trick}.
%As we have done the maths in another context for the dirichlet mixture, we get a similar result that ends up on the following fixed point algorithm:
%$$
%\fbox{
%    \displaystyle\left(\alpha_D^{(j)}\right)^{(m+1)} = \psi^{-1}\left(\frac{1}{n} \sum_{i=1}^n \log \left[X_i^{(D)}\right]_j + \psi \left(\sum_{r=1}^U \left(\alpha_{D}^{(r)}\right)^{(m)}\right) \right)
%}
%$$

We now have a way to compute $\theta^*$ as a MLE of the parameters of the joint distribution $p_{\theta}(X, T)$.

\subsubsection{Experiments}

We have implemented the previous modelization and tested it on a large microbiota dataset.

\subsubsection{Conclusions}

This first model is interesting as it provides a benchmark baseline for our upcoming tree-structured models.
However, it clearly lacks of complexity:
\begin{itemize}
    \item The branches are modeled as independent, which prevents any modelisation of correlation between entities.
          As for many ecosystems, we would expect some bacteria to have symbiotic relationship, or domination roles,
          especially when we come in critical systems like disease detection in which some bacteria may proliferate over others.
    \item The abundance data generation takes the tree constraints into account, yet it is poorly flexible as we sample from a unique Dirichlet
          parameterized by $\alpha_D$ and masked according to $T^{(D)}$, which completely omits the tree's global structure.
          It would be highly interesting to try to group the trees through a latent variable for instance, and adapt the parameter
          to that specific group to enhance the modelization.
          We will explore that situation in the next section.
\end{itemize}
\subsection{Taxonomy clustering: mixture model}

\subsubsection{Context}

A first natural extension to improve the model introduced in \ref{simple_generative_model} would be to take into account
a latent clustering of the microbiota.
Indeed, before looking at functional areas, it is highly likely that some people may share a similar taxonomic architecture. \\

Introducing that latent clustering can be done by modelizing a hidden variable that we will generally denote as $Z$.
We then define the following framework, which stands as a mixture of the previously introduced model \ref{simple_generative_model} over the latent clusters:
\begin{itemize}
    \item Denote $Z \in \{1, \dots, C\}$ the latent taxonomic clustering variable.
    \item Denote $p(Z = c) = \gamma_c$ the probability of cluster $c$.
    \item Assume that $T$ and $X$ distributions parameters are now conditioned by the value of $Z$:
          $$
          \begin{align}
                p(Z = c) &= \gamma_c \\
                p\left(u_k^{(\ell)} = 1 | \nodeparent(u_k^{(\ell)}) = 1, Z = c\right) &= \pi_{c,k}^{(l)} \\
                p\left(\childrennode(x_k^{(\ell)}) | x_k^{(\ell)}, T^{(\ell+1)}, Z = c\right) &\sim x_k^{(\ell)} \mathcal{D}(\alpha_{c,k}^{(l)} \odot T^{(\ell+1)}_k) \quad \text{if } |\childrennode(x_k^{(\ell)})| > 1
          \end{align}
          $$
\end{itemize}

\subsubsection{Optimization}

The usual optimization objective for this model would be the maximum likelihood regarding the observations, such that:
$$
\theta^* = arg\max_{\theta} p_{\theta}(T, X)
$$

However, this does not implement the latent structure that we are aiming for, as we do not observe $Z$.
Hence, using the Expectation-Maximization algorithm presented in \ref{section:EM}, we introduce a suboptimal objective on the
complete joint distribution:
$$
\theta^* = arg\max_{\theta} \underbrace{\mathbb{E}_{p_{\widehat{\theta}}(Z|T,X)}\left[ \log p_{\theta}(T, X, Z) | T, X \right]}_{Q(\widehat{\theta}, \theta)}
$$


\subsection{Hidden Markov Models: functional modelization}

\subsubsection{Context of Hidden Markov Models and Hidden Tree Markov Models}

One key goal of our analysis of the microbiota is to highlight functional groups of
bacteria in a studied context (reaction to treatment, overall view, disease tracking, post-surgery reaction, ...).
Unveiling such groups can be done through various processes, namely by using latent variables.

\medskip

Previously, we have worked with i.i.d latent variables, meaning that the hidden representation does not
implement any typical graph structure that would model latent dependencies between the bacteria (i.e., functional areas).
A typical latent model that would admit the simplest graph structure of a chain is given by the Hidden Markov Models.

\medskip

In a nutshell, Hidden Markov Models (HMM) are a class of generative models such that the latent variable follows a Markov Chain.
Formally, for an observed variable $Y = (Y_1, \dots, Y_n)$, $Z = (Z_1, \dots, Z_n)$ is a discrete HMM relatively to $Y$ if we have the following:
\begin{itemize}
    \item $Z$ is a Markov Chain (MC) with discrete states:
         $$
         \mathbb{P}(Z_{i+1} | Z_i, Z_{i-1}, \dots, Z_1) = \mathbb{P}(Z_{i+1} | Z_i)
         $$
    \item We define a prior on the seed of the MC:
         $$\forall c \in \{1, \dots, C\}, \mathbb{P}(Z_1 = c) = \pi_c$$

    \item We define a transition probability on the states of the MC:
        $$\forall c,k \in \{1, \dots, C\}^2, \mathbb{P}(Z_{i+1} = c | Z_i = k) = t_{k,c}^{(i+1)}$$

    \item We define an emission probability for $Y$ to be in a given state when observing $Z$:
        $$
        P(Y_i \in A | Z_i = c) = a_c(A)
        $$
\end{itemize}

\begin{figure}[H]
    \centering
    \includegraphics[scale=.25]{images/graph_dependency_HMM}
    \caption{Dependency graph of a HMM ($Z$ is the latent MC, $Y$ the observation)}
    \label{fig:graph_dependency_hmm}
\end{figure}

As a result, we can use HMM to model hidden functional groups of bacteria from one layer to the other.

This idea can even be extended to the specific tree-like structure of taxa-abundance data, introducing
the wider class of models given by the Hidden Tree Markov Model (HTMM) \cite{hidden_tree_markov_models}.
The HTMM provides a dependency graph that accounts for the tree structure, so that instead of a chain, we get
a Markov tree.
The following figure illustrates a given HTMM, precisely a top-down architecture:

\begin{figure}[H]
    \centering
    \includegraphics[scale=.5]{images/top_down_htmm_example}
    \caption{Top-down HTMM example: $Z$ represents the latent Markov tree,
        $X$ the observed abundance of a given bacteria in the taxonomic tree}
    \label{fig:top_down_htmm_example}
\end{figure}

The mathematical framework is then slightly different for top-down HTMM:
\begin{itemize}
    \item $Z$ is a Markov Tree (MT):
        $$\mathbb{P}\left(Z_{k}^{(l)} = c | Z_{0}^{(\ell)}, \dots, Z_{K_{\ell}}^{(\ell)}, \nodeparent(Z_{0}^{(\ell)}), \dots, \nodeparent(Z_{K_{\ell}}^{(\ell)}), \dots, Z_1^{(1)} \right) = \mathbb{P}\left(Z_{k}^{(\ell)} | \nodeparent(Z_{k}^{(\ell)})\right)$$
    \item We define a prior on the first layer of the MT, relatively to a given tree $T$:
        $$
        \mathbb{P}\left(Z_1^{(1)} = j | T^{(1)}\right) = a_j
        $$
    \item We define the transition probabilities of the MT, relatively to a given tree $T$:
        $$\mathbb{P}\left(\childrennode(Z_k^{(\ell)}) = \nu | Z_k^{(\ell)} = j, T^{(\ell+1)}\right) = t_{\nu,j}^{(\ell)}$$
\end{itemize}

Notice that this model would be tree dependent, as we want to attach existing nodes to a given hidden state, to generate
the abundance accordingly.
One could think of a tree generator model by applying the HTMM onto the global taxonomy directly, and defining an "empty"
state that would turn off the specific nodes in the final representation.

\medskip

Some recent work on HTMM have been showing that top-down modelling captures
less discriminent detail and conditional dependencies between the nodes than the bottom-up architecture \cite{bottom_up_superiority_hidden_tree_markov_models}.
In this first study though, we will focus on top-down models as a first try, as it makes the generation more natural (since we can not generate from bottom to up for the abundance).
Note that bottom-up modeling is still possible, but we can not generate sequentially the abundance with $Z$.
We would need to generate $Z$ entirely, and then generate $X$ in a top-down fashion.

\subsubsection{Tree generation: layer-wise Hidden Markov Model}

In this section, we aim at modeling the distribution of the trees $(T_i)_{1 \leq i \leq n}$ through a layer-wise Hidden Markov Model.
The following figure illustrates the dependency between the latent markov chain and the tree:
\begin{figure}[H]
    \centering
    \includegraphics[scale=.4]{images/layer_wise_hidden_markov_model_tree}
    \caption{Layer-wise hidden markov model dependency graph: $Z$ is the HMM, $T^{(\ell)}$ is the layer $(\ell)$ of the tree}
    \label{fig:layer_wise_hmm_tree}
\end{figure}

As illustrated on figure \ref{fig:layer_wise_hmm_tree} and by nature of our taxonomic data,
the layer of the tree follow at least a Markov process (it could be deeper than markovian).
Conversely to the previous situations, the latent variable now follows a markov chain as well, which enables to
group the layers of the tree with a dependence on the parent's layer group.

\medskip

Furthermore, in that approach, we decide to model the layers rather than the nodes to introduce dependency between
the nodes of a given layer, so we do not look into the $u_k^{(\ell)}$ anymore but directly into the $T^{(\ell)}$.

\medskip

Concisely, the mathematical framework can be written as follow:
\begin{itemize}
    \item $Z$ follows a discrete MC:
        $$
        \begin{align}
            &\mathbb{P}\left(Z^{(1)} = c\right) = a_c \\
            &\mathbb{P}\left(Z^{(\ell+1)} = k | Z^{(\ell)} = c\right) = t_{c,k}^{(\ell+1)}
        \end{align}
        $$
    \item $T$ is markovian and parameterized conditionally to $Z$.
          We denote by $\childrennode(T^{(\ell)})$ the set of plausible activation vectors at layer $(\ell+1)$ in the taxonomy, relatively to
          the parents activation given by $T^{(\ell)}$. We then model the emission probability by:
        $$\mathbb{P}(T^{(\ell+1)} = \omega | Z^{(\ell+1)} = c, T^{(\ell)}) = \frac{\pi_c(\omega) \mathds{1}_{\omega \in \childrennode(T^{(\ell)})}}{\sum_{\nu \in \childrennode(T^{\ell})} \pi_c(\nu)}$$

\end{itemize}

One may notice that such modelisation on $T$ will require a large number of parameters.
Indeed, the set of plausible activation vectors at layer $(\ell+1)$ relatively to $T^{(\ell)}$ can be quite big, as for each activate node $u_k^{(\ell)}$ in $T^{(\ell)}$
we get $2^{|\childrennode(u_k^{(\ell)})|} - 1$ plausible activated vectors of its children ($-1$ comes from the fact that we would like to avoid missing values).
We denote by $\mathcal{S}(Z)$ the discrete state space of the hidden markov chain.
Hence, for a given tree $T$, we obtain a total number of parameters of:
$$
n_{\text{params}} = \sum_{\ell=1}^{L}\sum_{k=1}^{K_{\ell}} |\mathcal{S}(Z)| \times \left(2^{|\childrennode(u_k^{(\ell)})|} - 1\right)
$$

As usual for latent data models, variational methods come in handy for learning the model.
Hence, we provide an Expectation-Maximization algorithm to compute the optimal parameters of the prior over the tree
in proposition \ref{proposition:em_layer_wise_hmm_tree}.

\subsubsection{Abundance generation: Hidden Tree Markov Model}

In this section, we aim at modeling the distribution of the abundance data $(X_i)_{1 \leq i \leq n}$ conditionally
to its known corresponding trees $(T_i)_{1 \leq i \leq n}$, using a Hidden Tree Markov Model (HTMM).
The dependency graph is illustrated by the figure \ref{fig:top_down_htmm_example}, giving us the follow mathematical framework:

\begin{itemize}
    \item $Z$ is a discrete Markov Tree (MT):
        \begin{itemize}
            \item We define a prior over the root:
                $$\mathbb{P}(Z_1^{(1)} = c | T^{(1)}) = a_j$$
            \item We define the transition probabilities:
                $$\mathbb{P}(\childrennode(Z_k^{(\ell)}) = \nu | Z_k^{(\ell)} = j, T^{(\ell+1)}) = t_{j,\nu}^{(\ell)}$$
        \end{itemize}
    \item $X$ follows a Markov Tree structure.
        We define the emission probability of the corresponding abundance relatively to the hidden states and tree:
        $$p(\childrennode(x_k^{(\ell)}) | \childrennode(Z_k^{(\ell)}) = \nu, x_k^{(\ell)}, T^{(\ell+1)}) \sim x_k^{(\ell)} \mathcal{D}(\alpha_{\nu,k}^{(\ell)} \odot T^{(\ell+1)}_k)$$
\end{itemize}

The number of parameters of that model is quite large, as it depends on the size of the latent space state and the number of child node
of each parent node in the tree structure.
Indeed, for each node, the corresponding dirichlet parameter is of the size of the number of children (if it has more than 1 child).
Then, we have as many dirichlet parameters as the possible amount of vector $\nu$, for which each coordinate has $|\mathcal{S}(Z)|$ plausible value.
That leads us to the following number of parameters:
$$
n_{\text{params}} = \sum_{\ell=1}^L \sum_{k=1}^{K_{\ell}} |\mathcal{S}(Z)|^{|\childrennode(u_k^{(\ell)})|} \times |\childrennode(u_k^{(\ell)})| \times \mathds{1}_{|\childrennode(u_k^{(\ell)})| > 1}
$$

As previously for the tree generation, we propose an Expectation-Maximization algorithm to compute the optimal parameters of
that HTMM, detailed in proposition \textbf{[TODO]}.

\subsubsection{Bottom-Up Hidden Tree Markov Model: general model}

The previous approach used a Top-Down HTMM (see figure \ref{fig:top_down_htmm_example}), which is a valid approach to model a parent to child dependency only.
Medically speaking, the top-down HTMM can be seen as parent dependent functions: a parent's function would determine the function of its child, which would be independently sorted conditionally to the parent.
In addition to that questionable modelisation, related work from \cite{bottom_up_superiority_hidden_tree_markov_models} as shown that top-down HTMM are capturing
less conditional depencencies between nodes, modeling less distributions than a bottom-up approach. \\

Hence, we are now interested in a bottom-up HTMM, which is illustrated in figure \ref{fig:bottom_up_htmm_example}.

\begin{figure}[H]
    \centering
    \includegraphics[scale=.4]{images/bottom_up_hidden_markov_tree_example}
    \caption{Bottom-down Hidden Tree Markov Model architecture: $Z$ represents the latent space, $X$ a taxa-abundance sample.
    The pointed out sections and arrows represent the dependencies.
    Each node is marked by its layer $(\ell)$ and position within that layer $k$.}
    \label{fig:bottom_up_htmm_example}
\end{figure}

\newcommand{\killingstate}{s_{\emptyset}}

The hidden states $Z$ are taking the graph dependency form of a tree that covers the entire taxonomy, and take value in a discrete set $\mathcal{S(Z)}$.
To take in consideration the sparsity of the taxa-abundance data, we introduce a killing state that we denote as $\killingstate$, which filter out a node and its children when appearing. \\

The general framework is then the following:
\begin{itemize}
    \item $Z$ is bottom-up HTMM with killing state $\killingstate$ so that:
            $$
            \begin{align}
                &p(Z^{(L)} = \nu) = \pi_{\nu} \\
                &p(Z_k^{(\ell)} = s | \childrennode(Z_k^{(\ell)}) = \nu) = t_{\nu, s}^{(\ell)}
            \end{align}
            $$
    \item $X$ is a top-down markov tree conditionally to $Z$:
            $$
            \begin{align}
                &p(X^{(1)}) = \delta_{e_1}(X^{(1)}) \\
                &p(X) = p(X^{(1)}) \prod_{\ell = 1}^{L-1} p(X^{(\ell+1)} | X^{(\ell)}) \\
                &p(X^{(\ell+1)} | X^{(\ell)}) = \prod_{k=1}^{K_{\ell}} p(\childrennode(x_k^{(\ell)}) | x_k^{(\ell)})
            \end{align}
            $$
    \item The conditional distribution of $X$ knowing $Z$ is characterised as follows:
            $$
            p(\childrennode(x_k^{(\ell)}) | x_k^{(\ell)}, \childrennode(z_k^{(\ell)}) = \nu) = x_k^{(\ell)} \mathcal{D}(\alpha_{k,\nu}^{(\ell)})
            $$
          $\alpha_{k,\nu}^{(\ell)}$ are the dirichlet parameters of the distribution, which are vectors of dimension $$\#\{j \in \nu | j \neq \killingstate\}$$ as they
          only impact the nodes that are still alive according to $Z$.
\end{itemize}

To compute the optimal parameters of that model given a dataset of taxa-abundance data $(X_i)_{1 \leq i \leq n}$, we propose an Expectation-Maximization algorithm
in proposition \ref{proposition:EM_bottom_up_HTMM_dirichlet}.
However, even if this model is covers all types of problems with its general form, it may not be suited to our special use case with very few samples.
Hence, the next section provides an under-parameterized alternative model to fit the framework of low samples datasets.

\subsubsection{Bottom-Up Hidden Tree Markov Model: low sample regime}

The previous bottom-up HTMM model is extremely general, as it provides a parameter for each transition configuration through $t^{(\ell)}_{\childrennode(Z_k^{(\ell)}), Z_k^{(\ell)}}$.
However, training such model onto a small dataset may be difficult and could result in overfitting, as we may highly likely not explore all possible configurations for a node and its children, and that all the configuration do not
impact each other's transition probabilities in the HTMM. \\

Hence, building on the previous model, we introduce a low parameterized model, which may not capture all the expressiveness of the general HTMM, but that could be trainable on a low sample regime.
\begin{itemize}
    \item Let $S \in \mathbb{N}$ the number of possible hidden states, at each layer $(\ell)$ we introduce the parameter $\pi^{(\ell)} \in \mathbb{R}^{S}$, which defines the aggregation parameter $\omega_k^{(\ell)}$:
            $$
            \omega_k^{(\ell)} = \sum_{z \in \childrennode(Z_k^{(\ell)})} \sum_{s=1}^{S} \mathds{1}_{z = e_s} \pi^{(\ell)}_s e_s
            $$
         where $e_s$ denotes the embedding of the state $s$ into a vector representation (one-hot encoding for instance).
    \item $Z$ is bottom-up HTMM with killing state $\killingstate$, for which the transition probabilities are defined through $\omega_k^{(\ell)}$:
    $$
    \begin{align}
        &Z_k^{(L)} \sim Cat(\pi^{(L)}) \\
        &Z_k^{(\ell)} | \childrennode(Z_k^{(\ell)}) \sim Cat(softmax(\omega_k^{(\ell)}))
    \end{align}
    $$
    \item $X$ is a top-down markov tree conditionally to $Z$:
    $$
    \begin{align}
        &p(X^{(1)}) = \delta_{e_1}(X^{(1)}) \\
        &p(X) = p(X^{(1)}) \prod_{\ell = 1}^{L-1} p(X^{(\ell+1)} | X^{(\ell)}) \\
        &p(X^{(\ell+1)} | X^{(\ell)}) = \prod_{k=1}^{K_{\ell}} p(\childrennode(x_k^{(\ell)}) | x_k^{(\ell)})
    \end{align}
    $$
    \item The conditional distribution of $X$ knowing $Z$ is characterised as follows, for the nodes that have at least $2$ children:
    $$
    p(\childrennode(x_k^{(\ell)}) | x_k^{(\ell)}, \childrennode(z_k^{(\ell)}) = \nu) = x_k^{(\ell)} \mathcal{D}(\alpha_{k}^{(\ell)} \odot \nu)
    $$
    $\alpha_{k}^{(\ell)}$ are the dirichlet parameters of the distribution, which are vectors of dimension $#\childrennode(Z_k^{(\ell)})$.
    The hadamard product with $\nu$ denotes that we mask the coordinates of $\alpha_k^{(\ell)}$ according to the states that are still alive i.e. different from $e_{\killingstate}$.
    If the node has $1$ child or none, the abundance value has to be the same as the parent.
\end{itemize}

Note that the hidden states are embedding vectors $e_s$ with $s \in \{1, \dots, S\}$.
So, for notation simplicity, we will introduce the following subscript notation.
Given $\theta$ a vector of dimension $S$, $Z$ a random variable taking value in the discrete state set $\mathcal{S}(Z)$ of dimension $S$, the embedding vectors of $\mathcal{S}(Z)$ are denoted by $e_s$:
$$
\theta_{Z} = \sum_{s=1}^{S} \mathds{1}_{Z=e_s} \theta_{s}
$$

Finding the optimal parameters of such model fitting a dataset can be done in the sense of the maximum likelihood, using the proposed Expectation-Maximization algorithm
of proposition ???.

\newpage
\section{Appendix}

\subsection{Microbiota analysis}

\begin{proposition}(Bernoulli tree prior)
    \label{proposition:bernoulli_tree_prior}
    \\
    Let $T = (T^{(1)}, \dots, T^{(L)})$ a random tree of depth $L$. \\
    For all $\ell \in \{1, \dots, L\}$, denote $T^{(\ell)} = (u_1^{(\ell)}, \dots, u_{K_{\ell}}^{(\ell)})$ the nodes at layer $\ell$,
    such that $u_k^{(\ell)} \in \{0, 1\}$ denotes whether or not the node is activated (1 if activated). \\
    Denote $\mathcal{P}(u_k^{(\ell)})$ the parent of the node $u_k^{(\ell)}$ that outputs $1$ if the parent exists and is activated, $0$ otherwise. \\
    Assume that:
    \begin{itemize}
        \item $p(T^{(1)}) = \delta_{e_1}(T^{(1)})$
        \item $\forall l \geq 2, p(T^{(\ell+1)} | T^{(1:\ell)}) = p(T^{(\ell+1)} | T^{(\ell)})$
        \item $\forall l \geq 2, k \in \{1, \dots, K_{\ell}\}, u_k^{(\ell)} | \left\{\mathcal{P}(u_k^{(\ell)}) = 1 \right\} \sim \mathcal{B}(\pi_k^{(\ell)}) \right$
        \item $\displaystyle p\left(u_1^{(\ell)}, \dots, u_{K_{\ell}}^{(\ell)} | \mathcal{P}(u_1^{(\ell)}), \mathcal{P}(u_{K_{\ell}}^{(\ell)})\right) = \prod_{k=1}^{K_{\ell}} p\left(u_k^{(\ell)} | \mathcal{P}(u_k^{(\ell)})\right)$
    \end{itemize}

    Then, the distribution of $T$ can be written as:
    $$
    p(T) = \delta_{e_1}(T^{(1)}) \prod_{l=1}^{L-1} \prod_{k=1}^{K_{\ell}} \left(\left(\pi_k^{(\ell+1)}\right)^{u_k^{(\ell+1)}} \left(1-\pi_k^{(\ell+1)}\right)^{1 - u_k^{(\ell+1)}} \right)^{\mathcal{P}(u_k^{(\ell+1)})}
    $$
\end{proposition}

\begin{proof}
    $$
    \begin{align}
        p(T) &= p\left(T^{(1)}, \dots, T^{(L)}\right) \\
        &= p\left(T^{(1)}\right) \prod_{l=1}^{L-1} p\left(T^{(\ell+1)}|T^{(\ell)}\right) \\
        &= \delta_{e_1}(T^{(1)}) \prod_{l=1}^{L-1} p\left(u_1^{(\ell+1)}, \dots, u_{K_{\ell}}^{(\ell+1)} | \nodeparent(u_1^{(\ell+1)}), \dots, \nodeparent(u_{K_{\ell}}^{(\ell+1)})\right) \\
        &= \delta_{e_1}(T^{(1)}) \prod_{l=1}^{L-1} \prod_{k=1}^{K_{\ell}} p\left(u_k^{(\ell+1)} | \nodeparent(u_k^{(\ell+1)}) \right) \\
        &= \delta_{e_1}(T^{(1)}) \prod_{l=1}^{L-1} \prod_{k=1}^{K_{\ell}} \left( \pi_k^{(\ell+1)} u_k^{(\ell+1)} + \left(1 - \pi_k^{(\ell+1)} \right)(1 - u_k^{(\ell+1)}) \right)^{\nodeparent(u_k^{(\ell+1)})} \\
        &= \delta_{e_1}(T^{(1)}) \prod_{l=1}^{L-1} \prod_{k=1}^{K_{\ell}} \left( \left(\pi_k^{(\ell+1)}\right)^{u_k^{(\ell+1)}} \left(1 - \pi_k^{(\ell+1)} \right)^{1 - u_k^{(\ell+1)}} \right)^{\nodeparent(u_k^{(\ell+1)})}
    \end{align}
    $$
\end{proof}

\begin{proposition}[MLE of the Bernoulli tree prior]
    \label{proposition:mle_bernoulli_tree_prior}
    \\
    Recall the context of \ref{proposition:bernoulli_tree_prior}. \\
    Let $(T_1, \dots, T_n)$ $n$ trees i.i.d following the bernoulli tree prior.
    Denote the maximum likelihood objective by
    $$
    \begin{equation}
        \begin{align}
            arg \max_{\pi_j^{(m)}} \quad & \sum_{i=1}^n \sum_{l=1}^{L-1} \sum_{k=1}^{K_{\ell}} \nodeparent(u_{k,i}^{(\ell+1)}) \left[u_{k,i}^{(\ell+1)} \log \pi_{k}^{(\ell+1)} + (1 - u_{k,i}^{(\ell+1)}) \log (1 - \pi_k^{(\ell+1)}) \right] \\
            \textrm{s.t.} \quad & \forall k,l, \pi_{k}^{(\ell)} \in [0, 1] \\
        \end{align}
        \label{eq:prior_transition_objective}
    \end{equation}
    $$

    Then the maximum likelihood estimator is given by
    $$
    \left(\pi_k^{(\ell)}\right)^* = \frac{\sum_{i=1}^n \nodeparent(u_{k,i}^{(\ell)}) u_{k,i}^{(\ell)}}{\sum_{i=1}^n \nodeparent(u_{k,i}^{(\ell)})}
    $$
\end{proposition}

\begin{proof}
    Simply by deriving the objective we obtain:
    $$
    \begin{align}
        \partial_{\pi_j^{(m)}} \log p(T_1, \dots, T_n) &= \sum_{i=1}^n \nodeparent(u_{j,i}^{(m)}) \left[ u_{j,i}^{(m)} \frac{1}{\pi_j^{(m)}} + (1 - u_{j,i}^{(m)}) \frac{-1}{1-\pi_j^{(m)}}\right] \\
        &= \frac{1}{\pi_j^{(m)}} \sum_{i=1}^n \nodeparent(u_{j,i}^{(m)}) u_{j,i}^{(m)} - \frac{1}{1 - \pi_j^{(m)}} \sum_{i=1}^n \nodeparent(u_{j,i}^{(m)}) (1 - u_{j,i}^{(m)})
    \end{align}
    $$

    Looking for $0$ valued gradient, we end up with:
    $$
    \begin{align}
    (1 - \pi_j^{(m)}) \sum_{i=1}^n \nodeparent(u_{j,i}^{(m)}) u_{j,i}^{(m)} - \pi_j^{(m)} \sum_{i=1}^n \nodeparent(u_{j,i}^{(m)}) (1 - u_{j,i}^{(m)}) &= 0 \\
    \pi_j^{(m)} \sum_{i=1}^n \nodeparent(u_{j,i}^{(m)}) &= \sum_{i=1}^n \nodeparent(u_{j,i}^{(m)}) u_{j,i}^{(m)} \\
    \pi_j^{(m)} &= \frac{\sum_{i=1}^n \nodeparent(u_{j,i}^{(m)}) u_{j,i}^{(m)}}{\sum_{i=1}^n \nodeparent(u_{j,i}^{(m)})}
    \end{align}
    $$

    Since the obtained value respects the constraint, that concludes the proof.
\end{proof}

\begin{proposition}[Law of $aX$]
    \label{proposition:law_of_aX}
    \\
    Let $X \sim f(x)$ a random variable with density $f$. \\
    Let $a \in (0, 1)$. \\
    The distribution of $aX$ is given by:
    $$p_{aX}(u) = \frac{1}{a} f\left(\frac{u}{a}\right)$$
\end{proposition}

\begin{proof}
    We use the transfer theorem.
    Let $g$ a continuous and measurable function.
    $$
    \begin{align}
        \mathbb{E}[g(aX)] &= \int g(ax) f(x) dx \\
                          &= \int g(u) f\left(\frac{u}{a}\right) \frac{1}{a} du
    \end{align}
    $$
    Hence, one can identify the distribution of $aX$ as $p_{aX}(u) = \frac{1}{a} f\left(\frac{u}{a}\right)$
\end{proof}

\begin{proposition}[Abundance distribution conditionally to the trees]
    \label{proposition:abundance_posterior_bernoulli_tree}

    \newline

    Let $X = (X^{(1)}, \dots, X^{(L)})$ the abundance matrix of a tree $T$ with depth $L$. \\
    For all $\ell \in \{1, \dots, L\}$, denote $X^{(\ell)} = (x_1^{(\ell)}, \dots, x_{K_{\ell}}^{(\ell)})$ the abundance value of each node of the tree. \\
    Denote $\childrennode(x_k^{(\ell)})$ the vector of children abundances related to $x_k^{(\ell)}$, which is empty if it has no children. \\
    Denote $T^{(\ell + 1)}_k$ the vector of activation states $(u_j^{(\ell+1)})_j$ of the children of the node $u_k^{(l)}$ in $T$. \\
    Assume that:
    \begin{itemize}
        \item $p(X^{(1)}) = \delta_{e_1}(X^{(1)})$
        \item $p(X^{(\ell+1)} | X^{(1:\ell)}, T) = p(X^{(\ell+1)} | X^{(\ell)}, T^{(\ell+1)})$
        \item $p(X^{(\ell+1)} | X^{(\ell)}, T^{(\ell+1)}) = \displaystyle\prod_{k=1}^{K_{\ell}} p(\childrennode(x_k^{(\ell)}) | x_k^{(\ell)}, T^{(\ell+1)})$
        \item For all $l \in \{2, \dots, L\}, k \in \{1, \dots, K_{\ell}\}$,
              \begin{itemize}
                  \item If $|\childrennode(x_k^{(\ell)})| > 1$: $\childrennode(x_k^{(\ell)}) | x_k^{(\ell)}, T^{(\ell+1)} \sim x_k^{(\ell)} D(\alpha_k^{(\ell)} \odot T^{(\ell+1)}_k)$
                  \item If $|\childrennode(x_k^{(\ell)})| = 1$, $\childrennode(x_k^{(\ell)}) |  x_k^{(\ell)}, T^{(\ell+1)} \sim \delta_{x_k^{(\ell)}}\left(\childrennode(x_k^{(\ell)})\right)$
              \end{itemize}
    \end{itemize}

    Then, the distribution of $X$ conditionally to $T$ is given by:
    $$
        p(X|T) = \delta_{e_1}(X^{(1)}) \prod_{l=1}^{L-1} \prod_{k=1}^{K_{\ell}} \left(
    \delta_{x_k^{(\ell)}}\left(\childrennode(x_k^{(\ell)})\right)^{\mathds{1}_{|\childrennode(x_k^{(\ell)})| = 1}}
    \frac{1}{x_k^{(\ell)}} f_{\alpha_k^{(\ell)} \odot T^{(\ell+1)}_k} \left(\frac{\childrennode(x_k^{(\ell)})}{x_k^{(\ell)}}\right)^{\mathds{1}_{|\childrennode(x_k^{(\ell)})| > 1}}
                \right)
    $$
    
\end{proposition}

\begin{proof}
    $$
    \begin{align}
        p(X | T) &= p(X^{(1)}, \dots, X^{(L)} | T) \\
                &= p(X^{(1)} | T^{(1)}) \prod_{l=1}^{L-1} p(X^{(\ell+1)} | X^{(\ell)}, T^{({\ell}+1)}) \\
                &= \delta_{e_1}(X^{(1)}) \prod_{l=1}^{L-1} \prod_{k=1}^{K_{\ell}} p(\childrennode(x_k^{(\ell)}) | x_k^{(\ell)}, T^{(\ell+1)}) \\
                &= \delta_{e_1}(X^{(1)}) \prod_{l=1}^{L-1} \prod_{k=1}^{K_{\ell}} \left(
                                                                                \delta_{x_k^{(\ell)}}\left(\childrennode(x_k^{(\ell)})\right)^{\mathds{1}_{|\childrennode(x_k^{(\ell)})| = 1}}
                                                                                \left[\underbrace{\frac{1}{x_k^{(\ell)}} f_{\alpha_k^{(\ell)} \odot T^{(\ell+1)}_k} \left(\frac{\childrennode(x_k^{(\ell)})}{x_k^{(\ell)}}\right)}_{\text{proposition \ref{proposition:law_of_aX}}}\right]^{\mathds{1}_{|\childrennode(x_k^{(\ell)})| > 1}}
                                                                            \right)
    \end{align}
    $$
\end{proof}

\begin{proposition}[MLE of the abundance distribution a posteriori]
    \label{MLE_abundance_bernoulli_tree}
    \\
    Consider the context of proposition \ref{proposition:abundance_posterior_bernoulli_tree}. \\
    We consider a data set $(X_i, T_i)_{1 \geq i \geq n}$ of abundance matrix and trees. \\
    Denote by $\mathcal{V}(\alpha_k^{(\ell)}, T) = \left\{v \in \mathbb{N} | \left[ \alpha_k^{(\ell)} \odot T^{(\ell+1)}_k \right]_v \neq 0\right\}$
    the set of indexes so that the coordinate of $\alpha_k^{(\ell)}$ is not masked by $T^{(\ell+1)}$. \\
    The maximum likelihood estimator of the distribution characterising the abundance conditionally to the tree is then
    given by the following fixed point algorithm: \\

    $
    \forall l \in \{1, \dots, L\}, k \in \{1, \dots, K_{\ell}\}, v \in \{1, \dots, |\alpha_k^{(\ell)}|\}
    $,

    $$
    \alpha_{k,v}^{(\ell)} \leftarrow \psi^{-1} \left(\frac{\sum_{i=1}^n \mathds{1}_{|\childrennode(x_{k,i}^{(\ell)})| > 1} \mathds{1}_{v \in \mathcal{V}(\alpha_k^{(\ell)}, T_i)} \left[ \psi\left(\sum_{u \in \mathcal{V}(\alpha_k^{(\ell)}, T_i)} \alpha_{k,u}^{(\ell)}\right) + \log \frac{\left[\childrennode(x_{k,i}^{(\ell)})\right]_v}{x_{k,i}^{(\ell)}} \right]}
    {\sum_{i=1}^n \mathds{1}_{|\childrennode(x_{k,i}^{(\ell)})| > 1} \mathds{1}_{v \in \mathcal{V}(\alpha_k^{(\ell)}, T_i)}} \right) \right)
    $$
\end{proposition}

\begin{proof}
    The maximum likelihood objective can be written as:
    $$
    \begin{align*}
        arg \max_{\alpha_{j,v}^{(m)}} \quad & \sum_{i=1}^n \sum_{l=1}^{L-1} \sum_{k=1}^{K_{\ell}} \mathds{1}_{|\childrennode(x_{k,i}^{(\ell)})| > 1} \log f_{\alpha_k^{(\ell)} \odot T_{k,i}^{(\ell+1)}} \left(\frac{C(x_{k,i}^{(\ell)})}{x_{k,i}^{(\ell)}} \right) \\
    \end{align*}
    $$
    Using the Dirichlet distribution expression, we can write the following:
    \small
    $$
    \begin{align}
        \log f_{\alpha_k^{(\ell)} \odot T_{k,i}^{(\ell+1)}}\left(\frac{C(x_{k,i}^{(\ell)})}{x_{k,i}^{(\ell)}} \right) = \log \Gamma \left(\sum_{v \in \mathcal{V}(\alpha_k^{(\ell)}, T_i)} \alpha_{k,v}^{(\ell)}\right)
        - \sum_{v \in \mathcal{V}(\alpha_k^{(\ell)}, T_i)} \Gamma \left( \alpha_{k,v}^{(\ell)} \right)
        + \sum_{v \in \mathcal{V}(\alpha_k^{(\ell)}, T_i)} \left(\alpha_{k,v}^{(\ell)} - 1\right) \log \frac{\left[\childrennode(x_{k,i}^{(\ell)})\right]_v}{x_{k,i}^{(\ell)}}
    \end{align}
    $$
    \normalsize

    Writing $\psi$ the digamma function, the derivative of the objective relatively to a fixed $\alpha_{k,v}^{(\ell)}$ is given by:
    $$
    \partial_{\alpha_{k,v}^{(\ell)}} \log p(X|T) = \sum_{i=1}^n \mathds{1}_{|\childrennode(x_{k,i}^{(\ell)})| > 1} \mathds{1}_{v \in \mathcal{V}(\alpha_k^{(\ell)}, T_i)} \left[\psi\left(\sum_{u \in \mathcal{V}(\alpha_k^{(\ell)}, T_i)} \alpha_{k,u}^{(\ell)} \right) - \psi\left(\alpha_{k,v}^{(\ell)} \right) + \log \frac{\left[\childrennode(x_{k,i}^{(\ell)})\right]_v}{x_{k,i}^{(\ell)}}\right]
    $$

    Looking for $0$ valued gradient, we obtain the following equation:
    $$
    \psi\left(\alpha_{k,v}^{(\ell)}\right) = \frac{\sum_{i=1}^n \mathds{1}_{|\childrennode(x_{k,i}^{(\ell)})| > 1} \mathds{1}_{v \in \mathcal{V}(\alpha_k^{(\ell)}, T_i)} \left[ \psi\left(\sum_{u \in \mathcal{V}(\alpha_k^{(\ell)}, T_i)} \alpha_{k,u}^{(\ell)}\right) + \log \frac{\left[\childrennode(x_{k,i}^{(\ell)})\right]_v}{x_{k,i}^{(\ell)}} \right]}
                                                {\sum_{i=1}^n \mathds{1}_{|\childrennode(x_{k,i}^{(\ell)})| > 1} \mathds{1}_{v \in \mathcal{V}(\alpha_k^{(\ell)}, T_i)}} \right)
    $$

    Using the reference trick from \cite{dirichlet_digamma_trick}, one can compute an iterative fixed point algorithm to solve the previous equation,
    leading to:
    $$
    \alpha_{k,v}^{(\ell)} \leftarrow \psi^{-1} \left(\frac{\sum_{i=1}^n \mathds{1}_{|\childrennode(x_{k,i}^{(\ell)})| > 1} \mathds{1}_{v \in \mathcal{V}(\alpha_k^{(\ell)}, T_i)} \left[ \psi\left(\sum_{u \in \mathcal{V}(\alpha_k^{(\ell)}, T_i)} \alpha_{k,u}^{(\ell)}\right) + \log \frac{\left[\childrennode(x_{k,i}^{(\ell)})\right]_v}{x_{k,i}^{(\ell)}} \right]}
    {\sum_{i=1}^n \mathds{1}_{|\childrennode(x_{k,i}^{(\ell)})| > 1} \mathds{1}_{v \in \mathcal{V}(\alpha_k^{(\ell)}, T_i)}} \right) \right)
    $$
\end{proof}


\begin{proposition}[Expectation-Maximization for Taxonomy Clustering model]
    \label{proposition:EM_taxonomy_clustering}
    \newline
    We use the same context as for proposition \ref{proposition:abundance_posterior_bernoulli_tree} of this section.
    Assume the following framework changes:
    \begin{itemize}
        \item $(X_i, T_i)_{1 \le i \le n}$ samples of taxa-abundance data ($X_i$ abundance, $T_i$ associated taxonomy)
        \item Let $Z_i$ a latent variable taking values in $\{1, \dots, C\}$.
        \item We model the distributions as follow:
            $$
            \begin{align}
                p(Z_i = c) &= \gamma_c \\
                p\left(u_{k,i}^{(\ell)} = 1 | \nodeparent(u_{k,i}^{(\ell)}) = 1, Z_i = c\right) &= \pi_{c,k}^{(\ell)} \\
                p\left(\childrennode(x_{k,i}^{(\ell)}) | x_{k,i}^{(\ell)}, T^{(\ell+1)}, Z_i = c \right) &\sim x_k^{(\ell)} \mathcal{D}(\alpha_{c,k}^{(\ell)} \odot T_k^{(\ell+1)}) \quad \text{if } |\childrennode(x_k^{(\ell)})| > 1
            \end{align}
            $$
    \end{itemize}

    Consider the optimization problem:
    $$
    \theta^* = arg\max_{\theta} \mathbb{E}_{p_{\widehat{\theta}}}[\log p_{\theta}(X, T, Z) | X, T]
    $$

    Then, the Expectation-Maximization algorithm solving that problem is given by (for each sample $i$ and cluster index $c$): \\

    \textbf{Expectation step}:

        $$
        p_{\widehat{\theta}}(Z_i = c | X_i, T_i) = \tau_{ic} =
                        \frac{\widehat{\gamma}_c p_{\widehat{\theta}}(T_i | Z_i = c) p_{\widehat{\theta}}(X_i | T_i, Z_i = c)
                             }
                             {
                                \sum_{c'=1}^C
                                \widehat{\gamma}_{c'} p_{\widehat{\theta}}(T_i | Z_i = c') p_{\widehat{\theta}}(X_i | T_i, Z_i = c')
                             }
        $$

        where:
        $$
        \begin{align}
            &p_{\widehat{\theta}}(T_i | Z_i = c) = \delta_{e_1}(T_i^{(1)}) \prod_{\ell = 1}^{L-1} \prod_{k=1}^{K_{\ell}} \left[\left(\widehat{\pi}_{c,k}^{(\ell+1)}\right)^{u_{k,i}^{(\ell+1)}} \left(1 - \widehat{\pi}_{c,k}^{(\ell+1)}\right)^{1 - u_{k,i}^{(\ell + 1)}}\right]^{\nodeparent(u_{k,i}^{(\ell+1)})} \\
            &p_{\widehat{\theta}}(X_i | T_i, Z_i = c) = \delta_{e_1}(X_i^{(1)}) \prod_{\ell = 1}^{L-1} \prod_{k=1}^{K_{\ell}} \delta_{x_{k,i}^{(\ell)}}\left(\childrennode(x_{k,i}^{(\ell)})\right)^{\mathds{1}_{|\childrennode(x_{k,i}^{(\ell)})| = 1}} \left(\frac{1}{x_{k,i}^{(\ell)}} f_{\widehat{\alpha}_{c,k}^{(\ell)} \odot T_{i,k}^{(\ell+1)}}\left(\frac{\childrennode(x_{k,i}^{(\ell)})}{x_{k,i}^{(\ell)}} \right) \right)^{\mathds{1}_{|\childrennode(x_{k,i}^{(\ell)})| > 1}}
        \end{align}
        $$

    \medskip
    \textbf{Maximization step}:

    \medskip

    $\displaystyle
    \gamma_c^*  = \frac{1}{n} \sum_{i=1}^n \tau_{ic}
    $

    $\displaystyle
        \left(\pi_{c,k}^{(\ell)}\right)^* &= \frac{\sum_{i=1}^n \nodeparent(u_{k,i}^{(\ell)}) u_{k,i}^{(\ell)} \tau_{ic}}{\sum_{i=1}^n \nodeparent(u_{k,i}^{(\ell)}) \tau_{ic}} \\
    $

    $\displaystyle
    \left[\alpha_{c,k}^{(\ell)}\right]_v &\leftarrow \psi^{-1} \left(\frac{\sum_{i=1}^n \mathds{1}_{|\childrennode(x_{k,i}^{(\ell)})| > 1} \mathds{1}_{v \in \mathcal{V}(\alpha_{c,k}^{(\ell)}, T_i)} \left[ \psi\left(\sum_{u \in \mathcal{V}(\alpha_{c,k}^{(\ell)}, T_i)} \left[\alpha_{c,k}^{(\ell)}\right]_u\right) + \log \frac{\left[\childrennode(x_{k,i}^{(\ell)})\right]_v}{x_{k,i}^{(\ell)}} \right]\tau_{ic}}
    {\sum_{i=1}^n \mathds{1}_{|\childrennode(x_{k,i}^{(\ell)})| > 1} \mathds{1}_{v \in \mathcal{V}(\alpha_{c,k}^{(\ell)}, T_i)} \tau_{ic}} \right) \right)
    $
\end{proposition}

\begin{proof}

    We study the two parts of the EM algorithm, starting with the expectation step.

    \medskip

    \textbf{Expectation step}:

    \medskip

    Applying Bayes formula, we obtain:
    $$
    \begin{align}
        p_{\widehat{\theta}}(Z_i = c | X_i, T_i) &= \tau_{ic} \\
                              &= \frac{p_{\widehat{\theta}}(Z_i = c, X_i, T_i)}{\sum_{c'=1}^C p_{\widehat{\theta}}(Z_i = c', X_i, T_i)} \\
                              &= \frac{\widehat{\gamma}_c p_{\widehat{\theta}}(T_i | Z_i = c) p_{\widehat{\theta}}(X_i | T_i, Z_i = c)}{\sum_{c'=1}^C \widehat{\gamma}_{c'} p_{\widehat{\theta}}(T_i = Z_i = c') p_{\widehat{\theta}}(X_i | T_i, Z_i = c')}
    \end{align}
    $$

    Exploiting the model framework, we obtain the following expression for the conditional distribution of the taxonomy:
    $$
    \begin{align}
        p_{\theta}(T_i | Z_i = c) &= p_{\theta}(T_i^{(1)}, \dots, T_i^{(L)} | Z_i = c) \\
                        &= \delta_{e_1}(T_i^{(1)}) \prod_{\ell = 1}^{L-1} p_{\theta}(T_i^{(\ell + 1)} | T_i^{(\ell)}, Z_i = c) \\
                        &= \delta_{e_1}(T_i^{(1)}) \prod_{\ell = 1}^{L-1} \prod_{k=1}^{K_{\ell}} p_{\theta}(u_{k,i}^{(\ell+1)} | \nodeparent(u_{k,i}^{(\ell+1)}), Z_i = c) \\
                        &= \delta_{e_1}(T_i^{(1)}) \prod_{\ell = 1}^{L-1} \prod_{k=1}^{K_{\ell}} \left[\left(\pi_{c,k}^{(\ell+1)}\right)^{u_{k,i}^{(\ell+1)}} \left(1 - \pi_{c,k}^{(\ell+1)}\right)^{1 - u_{k,i}^{(\ell + 1)}}\right]^{\nodeparent(u_{k,i}^{(\ell+1)})}
    \end{align}
    $$

    Similarly, the conditional abundance distribution is given by:

    $$
    \begin{align}
        p_{\theta}(X_i | T_i, Z_i = c) &= p_{\theta}(X_i^{(1)}, \dots, X_i^{(L)} | T_i, Z_i = c) \\
                                        &= \delta_{e_1}(X_i^{(1)}) \prod_{\ell = 1}^{L-1} p_{\theta}(X_i^{(\ell+1)} | X_i^{(\ell)}, T_i^{(\ell+1)}, Z_i = c) \\
                                        &= \delta_{e_1}(X_i^{(1)}) \prod_{\ell = 1}^{L-1} \prod_{k=1}^{K_{\ell}} p_{\theta}(\childrennode(x_{k,i}^{(\ell)}) | x_{k,i}^{(\ell)}, T_i^{(\ell+1)}, Z_i=c) \\
                                        &= \delta_{e_1}(X_i^{(1)}) \prod_{\ell = 1}^{L-1} \prod_{k=1}^{K_{\ell}} \delta_{x_{k,i}^{(\ell)}}\left(\childrennode(x_{k,i}^{(\ell)})\right)^{\mathds{1}_{|\childrennode(x_{k,i}^{(\ell)})| = 1}} \left(\frac{1}{x_{k,i}^{(\ell)}} f_{\alpha_{c,k}^{(\ell)} \odot T_{i,k}^{(\ell+1)}}\left(\frac{\childrennode(x_{k,i}^{(\ell)})}{x_{k,i}^{(\ell)}} \right) \right)^{\mathds{1}_{|\childrennode(x_{k,i}^{(\ell)})| > 1}}\\
    \end{align}
    $$

    \medskip

    \textbf{Maximization step}

    \medskip

    Recall the general objective:
    $$
    \begin{align}
        \theta^* &= arg\max_{\theta} Q(\widehat{\theta}, \theta) \\
                 &= arg\max_{\theta} \frac{1}{n} \sum_{i=1}^n \sum_{c=1}^C \left(\log \gamma_c + \log p(T_i | Z_i = c) + \log p(X_i | T_i, Z_i = c)\right) \tau_{ic}
    \end{align}
    $$

    We start by optimizing $\gamma$.
    The objective can be written as:
    $$
    \begin{equation}
        \begin{align}
            arg \max_{\gamma_j} \quad & \sum_{i=1}^n \sum_{c=1}^{C} \tau_{ic} \log \gamma_c  \\
            \textrm{s.t.} \quad & \sum_{c=1}^{C} \gamma_c = 1 \\
                                & \forall c, \gamma_c \geq 0
        \end{align}
        \label{eq:objective_latent_gamma_c_microbiota_clustering}
    \end{equation}
    $$

    This optimization problem can be solved using Lagrange duality.
    KKT conditions are then written as (optimizing relatively to any $\gamma_c$):
    $$
    \begin{align}
        &\frac{1}{\gamma_c} \sum_{i=1}^n \tau_{ic} + \phi - \psi = 0 \\
        &\sum_{c=1}^C \gamma_c = 1 \\
        &\psi \geq 0 \\
        &\forall c, \gamma_c \geq 0 \\
        &\psi \gamma_c = 0
    \end{align}
    $$

    Using the first identity, we get:
    $$
    \gamma_c = \frac{1}{\psi - \phi} \sum_{i=1}^n \tau_{ic}
    $$

    Let's set $\phi = 0$, which enables use to ensure the last identity.
    Using the second identity, we then have:
    $$
    \begin{align}
        \sum_{c=1}^C \gamma_c = 1 &= \sum_{c=1}^C \frac{1}{\psi} \sum_{i=1}^n \tau_{ic} \\
                                    &= \frac{1}{\psi} \sum_{i=1}^n \sum_{c=1}^C \tau_{ic} \\
                                    &= \frac{1}{\psi} \sum_{i=1}^n 1 \\
                                    &= \frac{n}{\psi}
    \end{align}
    $$

    Hence, we get $\psi = n$, which leads us to the optimal $\gamma_c$ that verifies the KKT constraints:
    $$
    \gamma_c^* = \frac{1}{n} \sum_{i=1}^n \tau_{ic}
    $$

    Looking at $\pi_c$ now, the optimization objective can be written as:
    $$
    \begin{equation}
        \begin{align}
            arg \max_{\pi_{c,k}^{(\ell)}} \quad & \sum_{i=1}^n \sum_{c=1}^{C} \tau_{ic} \log p_{\pi_c}(T_i | Z_i = c)  \\
            \textrm{s.t.} \quad & \pi_{c,k}^{(\ell)} \in [0, 1]
        \end{align}
    \end{equation}
    $$

    This objective is very similar to the MLE computed for proposition \ref{proposition:mle_bernoulli_tree_prior}, up to
    a multiplication by $\tau_{ic}$.
    Hence, we refer to the proposition \ref{proposition:mle_bernoulli_tree_prior} for the computation, which leads to the following
    optimal $\pi_c$:
    $$
    \left(\pi_{c,k}^{(\ell)}\right)^* = \frac{\sum_{i=1}^n \nodeparent(u_{k,i}^{(\ell))}) u_{k,i}^{(\ell)} \tau_{ic}}{\sum_{i=1}^n \nodeparent(u_{k,i}^{(\ell))}) \tau_{ic}}
    $$

    Similarly, the optimization objective of $\alpha_c$ is close to the one of proposition \ref{MLE_abundance_bernoulli_tree},
    up to a multiplication by $\tau_{ic}$.
    Therefore, we refer to \ref{MLE_abundance_bernoulli_tree} for the original computation, that just requires a multiplication by
    $\tau_{ic}$ to obtain the following result:
    $$
    \left[\alpha_{c,k}^{(\ell)}\right]_v &\leftarrow \psi^{-1} \left(\frac{\sum_{i=1}^n \mathds{1}_{|\childrennode(x_{k,i}^{(\ell)})| > 1} \mathds{1}_{v \in \mathcal{V}(\alpha_{c,k}^{(\ell)}, T_i)} \left[ \psi\left(\sum_{u \in \mathcal{V}(\alpha_{c,k}^{(\ell)}, T_i)} \left[\alpha_{c,k}^{(\ell)}\right]_u\right) + \log \frac{\left[\childrennode(x_{k,i}^{(\ell)})\right]_v}{x_{k,i}^{(\ell)}} \right]\tau_{ic}}
    {\sum_{i=1}^n \mathds{1}_{|\childrennode(x_{k,i}^{(\ell)})| > 1} \mathds{1}_{v \in \mathcal{V}(\alpha_{c,k}^{(\ell)}, T_i)} \tau_{ic}} \right) \right)
    $$

\end{proof}

\newpage
\section{Bibliography}
% \pagebreak
\nocite{*}
\printbibliography

\end{document}